\documentclass[11pt]{article}

\usepackage{amsmath}
\usepackage{amssymb}
\usepackage{pdflscape}

\begin{document}

\title{The Generation of Waves in Gravitational Fields by Orbiting Charges}

I present a case study of moving matter that generates gravitational  waves. The system I study consists of two charged point masses, slowly orbiting each other due to electromagnetic attraction, which source the gravitational fields.

This system is the easiest system that can be treated consistently within the framework of linearised theory and yields nontrivial results.

\section{The 3+1 Split}

To study my cases, I use the screen manifold formalism. Say the causal structure of the matter theory is given by the principal polynomial $P$

There are two relevant bases of the tangent space: the orthogonal basis $\{T, e_\alpha\}$ and the foliation basis $\{\dot{X} , \tilde{e}_\alpha\}$, with the corresponding cotangent space bases $\{ n, \epsilon^\alpha\}$ and $\{\kappa, \tilde{\epsilon}^\alpha \}$ respectively. 

The defining property of the orthogonal basis is

\begin{align}
	P \left( n , \dots, n \right) &:= 1 \label{frame_cond_1}\\ 
	P \left( \epsilon^\alpha , n,  \dots, n \right) &:= 0 \label{frame_cond_2}
\end{align}


The the two bases are related by

\begin{align}	
	\dot{X} &= N T + N^\alpha e_\alpha\\
	\tilde{e}_\alpha &= e_\alpha
\end{align}

and

\begin{align}	
	\kappa &= \frac{1}{N} n \\
	\tilde{\epsilon}_\alpha &= \epsilon_\alpha - \frac{1}{N} N^\alpha n
\end{align}

Formulas to derive / explain / proove:

\begin{align}
	\left[ e_{\alpha}, e_{\beta}\right] &= 0 \\
	\dot{T}^{A} &= \left( \mathcal{L}_{\dot{X}} T \right)^A
\end{align}

This section should contain a recipe which explains how to switch from the spacetime formalism to the screen manifold formalism for a generic quantity $T$ which comes with an action $S_T$:
\begin{enumerate}
	\item Determine the projections of $T$ on the screen manifold
	\item Determine the screen manifold action for those projections by expressing the spacetime action in foliation coordinates, and in terms of the projections
	\item From the action, calculate the Hamiltonian that generates the dynamics of the projections on the screen manifold with respect to the embedding parameter
\end{enumerate}


\section{Constructive Gravity / Grav. Closure of Matter Field Eq. }

The appropriate gravitational theory for a given matter theory depends on the causal structure (i.e. the principal polynomial $P$) of that matter theory, and is obtained by insisting on consistent dynamics. 

More precisely, a gravitational theory simply describes the dynamics of the causal structure of matter dynamics - it describes the dynamics of $P$.

$P$ is the only relevant geometric object; the Legendre map $L$, which maps vectors to covectors, is obtained from $P$ by

\begin{equation}
	L^a \left( k \right) = \frac{1}{\deg{P}}\frac{\mathrm{D}^a P \left( k \right)}{P\left(k\right)}
\end{equation}

The universal action of charged massive point particles in an electromagnetic field is

\begin{equation} \label{act_pp_univ}
	S = m \int \mathrm{d}\tau 
	\left(
		P 
		\left(
			L^{-1} 
			\left( 
				\frac{\mathrm{d} \gamma\left(\tau \right)}{\mathrm{d}\tau} 
			\right) 
		\right)^{\frac{1}{\deg{P}}}
		+ e A_a \frac{\mathrm{d} \gamma^a\left(\tau \right)}{\mathrm{d}\tau} 
	\right)
\end{equation}

\section{Orbiting Charges in Metric Spacetime}

In this section, I firstly derive the Hamiltonians of the electromagnetic field and of $N$ charged point particles. Then, I determine the motion of two charged point particles which interact via the electromagnetic field. Finally, I use the linearised equations of metric gravity to determine the gravitational waves generated by the matter system. 

Metric specific formulae:

\begin{align}
	g^{00} &= 1 \\
	g^{\alpha 0} &= 0 \\
	g^{\alpha \beta} &:= g_{(3)}^{\alpha \beta}  := - \gamma^{\alpha \beta} + \varphi^{\alpha \beta}\\
	g_{\alpha \beta} &= - \gamma^{\alpha \beta} - \varphi^{\alpha \beta}
\end{align}

\subsection{The Electromagnetic field} \label{metric_em}

This subsections contains the derivation of the Hamiltonian of the electromagnetic field. The laws of Maxwells electromagnetism are condensed in the action

\begin{align}
	S &= \frac{1}{16 \pi} \int \mathrm{d}^4 x \left( - \det{g^{\cdot \cdot}} \right)^{-\frac{1}{2}} F_{ab} F_{cd} \, \, g^{ac} g^{bd}\\
	F_{ab} &:= \partial_a A_b - \partial_b A_a 
\end{align}

which can be found in the book of Landau and Lifshitz.

On the screen manifold, the spacetime object $A_a$ decomposes into

\begin{align}
	\phi &:= T^a A_a\\
	A_{\alpha} &:= e^a_{\alpha} A_a
\end{align}

The derivative with respect to the embedding parameter $t$ is 

\begin{align}
	\dot{A}_{a} &:= \left( \mathcal{L}_{\dot{X}} A \right)_a\\
	\dot{\phi} &:= T^a \dot{A}_{a}\\
	\dot{A}_{\alpha} &:= e^a_{\alpha} \dot{A}_{a}
\end{align}


Now, I express $S$ in terms of fields $\phi$ and $A_{\alpha}$ on the screen manifold, their derivatives with respect to $y_\alpha$ and $t$, and lapse $N$ and shift $N^{\alpha}$. I find

\begin{align}
	F_{\alpha \beta} := e^a_{\alpha} e^b_{\beta} F_{a b} = F^{(3)}_{\alpha \beta}
\end{align}

where $F^{(3)}_{\alpha \beta} := \partial_{\alpha} A_{\beta} - \partial_{\beta} A_{\alpha}  $ and I used that $\left[ e_{\alpha}, e_{\beta}\right] = 0 $, and

\begin{align}
	F_{0 \beta} &:= T^a e^b_{\beta} F_{a b} = \frac{1}{N} \left( \dot{A}_{\beta} - \partial_{\beta} \left( N \phi +  N^{\alpha} A_{\alpha} \right) - N^{\alpha}  F^{(3)}_{\alpha \beta} \right)
\end{align}

using $ T = \frac{1}{N} \left( \dot{X} - N^\alpha e_{\alpha} \right)$. Kuchar  and Stone grant that 

\begin{align}
	\mathrm{d}^4 x \left( - \det{g^{\cdot \cdot}} \right)^{-\frac{1}{2}} = \mathrm{d}t \, \mathrm{d}^3 y N \left( - \det{g_{(3)}^{\cdot \cdot}} \right)^{-\frac{1}{2}}
\end{align}
 
By substituting the above expressions in $S$, I obtain an action for the fields $\phi (t, y) $ and $A_{\alpha} (t, y) $ on the screen manifold. 

\begin{align}
	S = \frac{1}{16 \pi} \int \mathrm{d}t \int \mathrm{d}^3 y \, \,
	&N \left( - \det{g_{(3)}^{\cdot \cdot}} \right)^{-\frac{1}{2}} \\ 
	\times 
	&\left(
	 2 g_{(3)}^{\alpha \beta} F_{\alpha 0 } F_{\beta 0 }  
	 + g_{(3)}^{\alpha \beta} g_{(3)}^{\gamma \delta } F^{(3)}_{\alpha \gamma}  F^{(3)}_{\beta \delta} 
	\right)
\end{align}

The embedding parameter $t$ controls the dynamics on the screen manifold and serves as the time parameter for the Hamiltonian treatment; the canonical momenta are 

\begin{align}
	\Pi^{\alpha} := \frac{\partial L}{\partial \dot{A}_{\alpha}} = \frac{1}{4\pi} \left( - \det{g_{(3)}^{\cdot \cdot}} \right)^{-\frac{1}{2}} 
	 g_{(3)}^{\alpha \beta} F_{0 \beta}
\end{align}

and the Hamiltonian is

\begin{align}
	H &:= \Pi^{\alpha} \dot{A}_{\alpha} - L = N \mathcal{H} + N^\alpha \mathcal{D}_{\alpha}\\
\end{align}

with 

\begin{equation}
\begin{split}
	\mathcal{H} :=  
	2 \pi \left( - \det{g_{(3)}^{\cdot \cdot}} \right)^{\frac{1}{2}} 
	& g_{\alpha \beta}^{(3)} \Pi^{\alpha} \Pi^{\beta} \\
	 - &\frac{1}{16 \pi} \left( - \det{g_{(3)}^{\cdot \cdot}} \right)^{- \frac{1}{2}} g_{(3)}^{\alpha \beta} g_{(3)}^{\gamma \delta } F^{(3)}_{\alpha \gamma}  F^{(3)}_{\beta \delta} 
	- \phi \partial_{\alpha} \Pi^{\alpha} 
\end{split}
\end{equation}

\begin{align}
	\mathcal{D}_{\alpha} &:= \Pi^{\beta} F^{(3)}_{\alpha \beta} -  A_{\alpha} 
	\partial_{\beta} \Pi^{\beta}
\end{align}


For scenarios including only weak gravitational fields, I obtain a decent approximation to this Hamiltonian by substituting  

\begin{align} 
	g_{(3)}^{\alpha \beta}  
	&:= - \gamma^{\alpha \beta} + \varphi^{\alpha \beta} \label{def_pert_met} \\
	N &:= 1 + A 
\end{align}
	
and then expanding the Hamiltonian to first order. I arrive at

\begin{equation} \label{mel_pert_ham}
\begin{split}
	H :=  
	& - \left( 1 + A \right) 
	\left[
	2 \pi \gamma_{\alpha \beta} \Pi^{\alpha} \Pi^{\beta}
	 + \frac{1}{8 \pi} \gamma_{\alpha \beta} H^{\alpha} H^{\beta}
	 + \phi \partial_{\alpha} \Pi^{\alpha}
	 \right]\\
	 & -  \varphi_{\mu \nu} 
	 \left( \delta_{\alpha}^{\mu}  \delta_{\beta}^{\nu} -  \frac{1}{2} \gamma^{\mu \nu } \gamma_{\alpha \beta} \right)
	 \left[
	 2 \pi  \Pi^{\alpha} \Pi^{\beta}
	 + \frac{1}{8 \pi} H^{\alpha} H^{\beta}
	 \right] \\
	 & + N^{\mu} \left[ 
	 \epsilon_{\mu \alpha \beta} \Pi^{\alpha} H^{\beta} -  A_{\mu} \partial_\alpha \Pi^\alpha
	 \right]
\end{split}
\end{equation}


where 

\begin{equation} \label{def_H_alpha}
	H^\alpha := \frac{1}{2}\epsilon^{\alpha \beta \gamma} F^{(3)}_{\alpha \beta} 
\end{equation}


\subsection{Point Particles} \label{sec_pp_met}

In relativistic theory, a point particle corresponds to a curve $\gamma$ in spacetime. To switch to the screen manifold formalism - where the particle is represented by a position $\lambda (t) $ which changes with the foliation parameter $t$ - I parametrise $\gamma$ with the foliation parameter $t$ and decompose the particles velocity (i.e. the vector tangent to the parametrised curve) according to

\begin{equation}
	\frac{\mathrm{d} \gamma (t)}{\mathrm{d} t } =: \dot{X} +  e_\alpha v^\alpha
\end{equation}

This decomposition is defined such that $v = 	\frac{\mathrm{d} \lambda (t)}{\mathrm{d} t } := \dot{\lambda} (t) $, so if $ v^\alpha = 0 $ the particle  moves with the embedding e.g. remains at the same spot on the screen manifold.

The next step towards the autonomous screen manifold formalism requires to express the action that governs the dynamics of the particle solely in screen manifold quantities. For metric geometry, the universal action for a point mass in the presence of the electromagnetic field reads

\begin{equation}
	S = \int \mathrm{d}\tau  \left[
	m \left( 
	g_{a b}\left( \gamma \right)
	\frac{\mathrm{d} \gamma^a\left( \tau \right) }{\mathrm{d}\tau}
	\frac{\mathrm{d} \gamma^b\left( \tau \right) }{\mathrm{d}\tau} \right)^{\frac{1}{2}}
	+ e A_a \left( \gamma  \right) 
	\frac{\mathrm{d} \gamma^a\left( \tau \right) }{\mathrm{d}\tau} 
	\right]
\end{equation}

Now parametrise $\gamma$ with $t$ and use the foliation frame to obtain the screen manifold action

\begin{equation}
\begin{split}
	S = \int \mathrm{d}\tau
	& m \left( 
	N^2 + g^{(3)}_{\alpha \beta} N^{\alpha} N^{\beta}
	+ 2 g^{(3)}_{\alpha \beta} v^\alpha N^\beta + g^{(3)}_{\alpha \beta} v^\alpha v^\beta 
	\right)^{\frac{1}{2}} \\
	+ \, & e \left( N \phi + N^\alpha A_\alpha + v^\alpha A_\alpha \right)
\end{split}
\end{equation}

To calculate the Hamiltonian of the point mass, I proceed slightly differently than above: this time, I firstly linearise the action and then execute the Legendre transform with the benefit that the inversion in favour of the velocity must not be done exactly, but pertubatively.

The linearised action reads

\begin{equation}
\begin{split}
	S = \int \mathrm{d}\tau  \, \, 
	& m \left(
	\left( 1 - v^2 \right)^{\frac{1}{2}}
	+ \left( 1 - v^2 \right)^{- \frac{1}{2}}
	\left( A - \gamma_{\alpha \beta} N^\beta v^\alpha - \frac{1}{2} \varphi_{\alpha \beta} v^\alpha v^\beta \right)
	\right)\\
	+ \, \, & e \left( \phi + v^\alpha A_\alpha + A \phi + N^\alpha A_\alpha \right) + \mathcal{O} \left(\varphi^2 \right)
\end{split}
\end{equation}

The canonical momentum of the point mass is

\begin{equation} \label{momentum_pp_met}
	p_\alpha := \frac{ \partial L}{\partial v^\alpha}
	= -m \frac{v^\alpha}{\sqrt{1- v^2}} + e A_\alpha	 
	+ \mathcal{O} \left(\varphi \right)
\end{equation}

I define $k_\alpha = p_\alpha - e A_\alpha$, expand $v^\alpha = v^\alpha_0 + v^\alpha_1 + \dots $ in orders of $\varphi$ and, by virtue of

\begin{align}
	k_\alpha &= -m \frac{v^\alpha}{\sqrt{1- v^2}} 
	 &\Leftrightarrow 
	 v_\alpha &= - \frac{k_\alpha}{\sqrt{k^2 + m^2}} 
\end{align}

I obtain

\begin{equation} \label{veloc_pp_met}
	v_\alpha = - \frac{k_\alpha}{E_k} + \mathcal{O} \left(\varphi \right)
\end{equation}

where $E_k := \sqrt{k^2+ m^2}$. Accommodatingly, knowledge of the zeroth order of $v^\alpha$ suffices to obtain the linearised Hamiltonian throught the Legendre transform:

\begin{equation} \label{ham_pp_met}
	H := p_\alpha v^\alpha - L  
	= - \left( 1 + A \right) \left( E_k + e \phi \right)
	- N^\alpha p_\alpha + \frac{1}{2 E_k} \varphi_{\alpha \beta} k^\alpha k^\beta + \mathcal{O} \left(\varphi^2 \right)
\end{equation}

where all occurring fields are evaluated at the position $\lambda$ of the particle.

\subsection{The Motion of Two Charged Point Masses} \label{sec_mot_bin}

The Hamiltonian of the complete system containing both charged point masses and electromagnetic fields is the sum of the respective Hamiltonians:

\begin{equation}
	H_{\text{Matter}}  = \sum_i H^{(i)}_{\text{Point particle}} + H_{\text{Electromagnetism}}
\end{equation}


This Hamiltonian is a screen manifold object, which generates the dynamics on the screen manifold with respect to the embedding parameter $t$.

The Hamiltonian contains the gravitational fields to linear order, so the resulting equations of motion describe the dynamics of electromagnetic fields and point masses in presence of weak gravitational fields. However - as I shall elaborate on later - to consistently analyse gravitational waves in linearised theory, the dynamics of the matter that generates those waves must happen on a flat background. Therefore, for the purpose of determining the equations of motion for the present matter and henceforth the dynamics of the source system, I shall neglect any gravitational fields. 

The Hamiltonian of one charged point mass and the electromagnetic field in a flat background is

\begin{equation}
\begin{split}
		H = &- \left(E_k + e \phi \left( \lambda \right) \right)\\
		&- \int \mathrm{d}^3 y \left( 2 \pi \gamma_{\alpha \beta} \Pi^\alpha \Pi^\beta + \frac{1}{8 \pi}\gamma_{\alpha \beta} H^\alpha H^\beta
		+ \phi \partial_\alpha \Pi^\alpha \right)
\end{split}
\end{equation}

Using the Hamiltonian field equations $\dot{A}_\alpha = \delta H / \delta \Pi^\alpha$ and $\dot{\Pi}^\alpha = -\delta H / \delta A_\alpha$,  the constraint equation $0 = \delta H / \delta \phi$ and the defining equation of $H^\alpha$,  I obtain Maxwells equations

\begin{align}
	\dot{E}_\alpha - \left(\partial \times H \right)_\alpha &= 4 \pi e 
	\dot{\lambda}_\alpha \delta_\lambda\\
		\dot{H}_\alpha +  \left( \partial \times E \right)_\alpha &= 0
		\label{eom_em_2} \\
	\partial_\alpha E^\alpha &= - 4 \pi e \delta_\lambda
	\label{eom_em_3} \\ 
	\partial_\alpha H^\alpha &= 0
\end{align}

for the electromagnetic fields on the screen manifold, where $E^\alpha := 4 \pi \Pi^\alpha = \partial_\alpha \phi - \dot{A}_\alpha $. Further, combining the Hamiltonian equations $\dot{q} = \partial H / \partial p$ and $\dot{p} = - \partial H / \partial q$ for the point mass, I obtain 

\begin{equation}
	\frac{\mathrm{d}}{\mathrm{d}t} 
	\left( \frac{m v_\alpha}{\sqrt{1 - v^2 }}\right)
	= -e \left[ \left(v \times H \right)_\alpha + E_\alpha \right] 
\end{equation} 

The force on the right hand side is the familiar Lorentz Force.

With the equations of motion at hand, I shall next present a particular solution, namely the slowly orbiting binary system. Slowly moving point masses move with velocities much less than the speed of light, e.g.  $ v \ll 1 $ in units where $c = 1$. For such masses, I shall therefore neglect all but the leading order contribution in $v$.

A point charge $e_1$ at the position $\lambda_1$, moving with a velocity $v_1$, generates an electric field 

\begin{equation}
	E^\alpha(y) = e_1 \frac{y^\alpha - \lambda_1^\alpha }{\left| y - \lambda_1\right|^3} 
	+ \mathcal{O}\left(v_1^2 \right)
\end{equation}

and a magnetic field

\begin{equation}
	H^\alpha = \mathcal{O}\left(v_1 \right)
\end{equation}

according to Landau and Lifshitz. A second particle at position $\lambda_2$ with charge $e_2$, mass $m_2$ and velocity $v_2$ exposed to the field of the first particle then obeys 

\begin{equation} \label{eom_pp_newt}
	m_2 \dot{v_2}^\alpha = e_1 e_2 \frac{\left( \lambda_1 - \lambda_2\right)^\alpha}{\left| \lambda_1 - \lambda_2\right|^3} + \mathcal{O}\left(v^2\right)
\end{equation}

Of course, the situation of the second particle is totally similar; I obtain an equation for its velocity simply by exchanging the labels $1$ and $2$ in eq. \ref{eom_pp_newt}. 

Defining $\lambda_{\text{rel}} = \lambda_1 - \lambda_2$ and $\mu = m_1 m_2 / \left( m_1 + m_2 \right)$, I rewrite the equations of motion for the system as

\begin{align}
	\mu \frac{\mathrm{d}^2}{\mathrm{d}t^2} \lambda_{\text{rel}}^\alpha 
	&= e_1 e_2 \frac{\lambda_{\text{rel}}^\alpha}{\left| \lambda_{\text{rel}} \right|^3} \label{eom_rel}\\
	\frac{\mathrm{d}^2}{\mathrm{d}t^2} \left( m_1 \lambda_1 + m_2 \lambda_2 \right)^\alpha 
	&= 0 \label{eom_com}
\end{align}

Eq. \ref{eom_com} allows the solution $m_1 \lambda_1 + m_2 \lambda_2 = 0$ which corresponds to a resting centre of mass.

The ansatz for circular motion

\begin{align}
	\lambda_{\text{rel}}  
	= d\begin{pmatrix}
	\cos{\omega t}\\ \sin{\omega t}\\ 0
	\end{pmatrix}
\end{align}

solves eq. \ref{eom_rel} for a frequencies

 \begin{equation}
 	 \omega = \pm \sqrt{- e_1 e_2 / \mu d^3} \label{freq_bin}
 \end{equation}
 
The positive frequency solution shall be referred to as $\omega_{\text{bin}}$.Therefore, a particular approximate solution (realistic as long as the velocities are very small) of the flat space equations of motion is given by

\begin{align}
	\lambda_1 &= \frac{\mu}{m_1}
	d\begin{pmatrix}
	\cos{\sqrt{- e_1 e_2 / \mu d^3} t}\\ \sin{\sqrt{- e_1 e_2 / \mu d^3} t}\\ 0
	\end{pmatrix}\\
	\lambda_2 &= \frac{\mu}{m_2}
	d\begin{pmatrix}
	\cos{\sqrt{- e_1 e_2 / \mu d^3} t}\\ \sin{\sqrt{- e_1 e_2 / \mu d^3} t}\\ 0
	\end{pmatrix}\\
	E^\alpha \left( y \right) &= 
	e_1 \frac{y^\alpha - \lambda_1^\alpha }{\left| y - \lambda_1\right|^3} 
	+ e_1 \frac{y^\alpha - \lambda_1^\alpha }{\left| y - \lambda_1\right|^3} 
	\label{E_bin}\\
	H^\alpha &= 0 \label{H_bin}
\end{align}

which shall henceforth be analysed with respect to it's ability to generate waves in the gravitational fields.


\subsection{The Generation of Gravitational Waves in Metric Spacetime}



\section{Orbiting Charges in Weakly Birefringent Spacetime}

How would gravity work if the electromagnetic field obeyed the equations of motion of birefringent electrodynamics? These equations feature a different causal structure than the familiar light cone of Maxwell electrodynamics, namely a double light cone. The appropriate gravitational theory, i.e. the  dynamics of this double cone, has been determined as part of the constructive gravity program.

In this section, I shall firstly implement the screen manifold formalism for birefringent electrodynamics. Then, I shall investigate the affiliated point particle theory, which also features the double cone as its causal structure. Finally, I shall investigate how the system I presented in sec. \ref{sec_mot_bin} generates waves in the gravitational fields that make up the causal structure.

\subsection{The Electromagnetic Field} \label{sec_em_am}

Birefringent electrodynamics is a generalisation of Maxwells electrodynamics: just as Maxwells theory, birefringent electrodynamics features a linear equation of motion and thus shares the superposition principle, but in contrast to Maxwell theory, it allows for birefringence in vacua.

In birefringent electrodynamics, the dynamics of the electromagnetic potential $A_a$ are controlled by the action 

\begin{equation} \label{act_gled}
	S = \frac{1}{32 \pi} \int \mathrm{d}^4 x \omega F_{a b} F_{c d} \, \,
	G^{a b c d}
\end{equation}

where $F_{a b} = \partial_a A_b - \partial_b A_a $ as usual, and 

\begin{equation}
	\omega = \left( \frac{1}{4 !} G^{a b c d} \epsilon_{a b c d} \right)^{-1}
\end{equation}

 $G^{a b c d}$ is called the area metric, and features the symmetries

\begin{equation}
	G^{a b c d} = G^{c d a b} = - G^{b a c d}	
\end{equation}

Implementing these symmetries in the expansion of $G$ in the orthogonal basis yields 

\begin{equation}
	\begin{split}
		G^{a b c d} 
		&= 4 G^{\beta \delta} T^{[a} e_\beta^{b]} T^{[c} e_\beta^{b]}\\
		&+ 2 G^{\beta \gamma \delta} T^{[a} e_\beta^{b]} e_\gamma^{c} e_\delta^{d}
		+ 2 G^{\alpha \beta \delta} e_\alpha^a e_\beta^{b} T^{[c} e_\delta^{d]}\\
		&+ G^{\alpha \beta \gamma \delta} e_\alpha^a e_\beta^{b]} e_\gamma^{c} e_\delta^{d}
	\end{split}   
\end{equation}

where

\begin{align}
	G^{\beta \delta} 
	&:= G \left( n, \epsilon^\beta, n, \epsilon^\delta \right)\\
	G^{\beta \gamma \delta} 
	&:= G \left( n, \epsilon^\beta, \epsilon^\gamma, \epsilon^\delta \right)\\
	G^{\alpha \beta \gamma \delta} 
	&:= G \left( \epsilon^\alpha, \epsilon^\beta, \epsilon^\gamma, \epsilon^\delta \right)
\end{align}

Now, I again undertake the construction of a screen manifold Hamiltonian that generates the same dynamics as the spacetime action \ref{act_gled}. The definition of the screen manifold fields $\phi$ and $A_\alpha$, as well as the expressions of $F_{ 0 \beta}$ and $F_{\alpha \beta}$ in terms of $N, N^\alpha, \dot{A}_\alpha$ and $F^{(3)}_{\alpha \beta}$, are the same as in sec. \ref{metric_em}

For later convenience, I shall at this point explicitly reduce the action \ref{act_gled} by a zero contribution. Since

\begin{equation}
	\int \mathrm{d}^4 x \epsilon^{a b c d} F_{a b} F_{c d} 
	= \int \mathrm{d}^4 x \epsilon^{a b c d} \partial_a A_b \partial_c A_d
	= - \int \mathrm{d}^4 x \epsilon^{a b c d} \partial_c \partial_a A_b A_d
	= 0
\end{equation}

using integration by parts, the contribution of the totally antisymmetric part of $G$ to the action vanishes and might as well be explicitly subtracted. The action then becomes

\begin{equation} 
	S = \frac{1}{32 \pi} \int \mathrm{d}^4 x \omega F_{a b} F_{c d} \, \,
	\left[ G^{a b c d} + \frac{1}{4!} \epsilon^{a b c d} G^{m n o p} \epsilon_{m n o p}
	\right]
\end{equation}

The weird $+$ sign arises due to convention \ref{conv_eps}. Expressed only in screen manifold fields and coordinates, the action becomes

\begin{equation} 
\begin{split}
	S = \frac{1}{32 \pi} \int \mathrm{d}^4 x \,\,\omega 
	[ 
	&4 G^{\beta \delta} F_{0 \beta} F_{0 \delta} \\
	&+ \left( 
	2 \left( G^{\beta \gamma \delta} + G^{ \gamma \delta \beta} \right)
	- \frac{4 N}{\omega} \epsilon^{\beta \gamma \delta}
	\right) F_{0 \beta} F_{\gamma \delta}\\
	&+ G^{\alpha \beta \gamma \delta} F_{\alpha \beta} F_{\gamma \delta}
	-  \frac{4 N^\mu}{\omega} 
	\epsilon^{\alpha \beta \gamma} F_{\mu \alpha} F_{\beta \gamma}
	]
\end{split}
\end{equation}

The canonical momentum is

\begin{equation}
	\Pi^\mu = \frac{\omega}{4 \pi N} 
	\left[ G^{\mu \nu} F_{0 \nu}
	+ \left( \frac{1}{2} G^{ \mu \alpha \beta} 
	- \frac{N}{2 \omega} \epsilon_{ \mu \alpha \beta}
	\right)
	F_{\alpha \beta}
	\right]
\end{equation}

I define 

\begin{equation}
	P^\alpha :=
	\frac{\omega}{4 \pi N} G^{\alpha \beta} F_{0 \beta}
	= \Pi^\alpha 
	- \frac{\omega}{4 \pi N} 
	\left( 
	\frac{1}{2} G^{\mu \alpha \beta} 
	- \frac{N}{2 \omega} \epsilon^{\mu \alpha \beta}
	\right)
	F_{\alpha \beta}
\end{equation}

and, through the usual Legendre transform, obtain the Hamiltonian 

\begin{equation}
	\begin{split}
		H &= N
		\left[
		\frac{2 \pi N}{\omega} G^{-1}_{\alpha \beta} P^\alpha P^\beta
		- \frac{\omega}{32 \pi N} G^{\mu \nu \alpha \beta} F_{\mu \nu} F_{\alpha \beta}
		- \phi \partial_\alpha \Pi^\alpha
		\right]\\
		&+ N^\mu
		\left[
		\frac{1}{8 \pi} \epsilon^{\alpha \beta \gamma} F_{\mu \alpha} F_{\beta \gamma} 
		+ F_{\mu \alpha} \Pi^{\alpha} - A_\mu \partial_\alpha \Pi^\alpha
		\right]
	\end{split}
\end{equation}

Within the community of constructive gravity, it is common not to work with the orthogonal frame components of $G$ directly, but rather with the equivalent set of fields

\begin{align}
	\bar{g}^{\alpha \beta} 
	&:= - G^{\alpha \beta}\\
	\bar{\bar{g}}_{\alpha \beta}
	&:= \frac{1}{4}\left( \det{\bar{g}^{\cdot \cdot}} \right)
	\epsilon_{\alpha \mu \nu} \epsilon_{\beta \rho \sigma}
	G^{\mu \nu \rho \sigma}\\
	{\bar{\bar{\bar{g}}}^\alpha}_\beta
	&:= \frac{1}{2}\left( \det{\bar{g}^{\cdot \cdot}} \right)^{-\frac{1}{2}}
	\epsilon_{\beta \mu \nu} G^{\alpha \mu \nu}
	 - \delta^\alpha_\beta
\end{align}

The frame conditions \ref{frame_cond_1} and \ref{frame_cond_2} translate into

\begin{align}
	{\bar{\bar{\bar{g}}}^\alpha}_\alpha  &= 0 \\
	{\bar{\bar{\bar{g}}}^\alpha}_\mu \bar{g}^{\mu \beta} 
	&= {\bar{\bar{\bar{g}}}^\beta}_\mu \bar{g}^{\mu \alpha} 
\end{align}

Of course, the components of $G$ might be reassembled from the fields $\bar{g}$, $\bar{\bar{g}}$ and $\bar{\bar{\bar{g}}}$ via

\begin{align}
	G^{\alpha \beta} &= - \bar{g}^{\alpha \beta} \\
	G^{\beta \gamma \delta} 
	&= \left( \det{\bar{g}^{\cdot \cdot}} \right)^{\frac{1}{2}}
	\epsilon^{\alpha \gamma \delta} 
	\left( 
	{\bar{\bar{\bar{g}}}^\beta}_\alpha 
	+ \delta^\beta_\alpha 
	\right) \\
	G^{\alpha \beta \gamma \delta } &= 
	\epsilon^{\alpha \beta \mu} \epsilon^{\gamma \delta \nu}
	\left( \det{\bar{g}^{\cdot \cdot}} \right) 
	\bar{\bar{g}}_{\mu \nu}
\end{align}

In terms of $\bar{g}$, $\bar{\bar{g}}$ and $\bar{\bar{\bar{g}}}$, the density $\omega$ reads

\begin{equation}
	\omega = N \left( \det{\bar{g}^{\cdot \cdot}}\right)^{-\frac{1}{2}}
\end{equation}

and the Hamiltonian becomes

\begin{equation}
	\begin{split}
		H =
		- N [
		&2 \pi \left( \det{\bar{g}^{\cdot \cdot}}\right)^{\frac{1}{2}}
		\bar{g}^{-1}_{\alpha \beta} 
		\left( 
		\Pi^\alpha 
		- \frac{1}{4 \pi}  {\bar{\bar{\bar{g}}}^\alpha }_\mu H^\mu
		\right)
		\left( 
		\Pi^\beta
		- \frac{1}{4 \pi}  {\bar{\bar{\bar{g}}}^\beta }_\nu H^\nu
		\right) \\
		&- \frac{1}{8 \pi} 
		\left( \det{\bar{g}^{\cdot \cdot}}\right)^{\frac{1}{2}}
		\bar{\bar{g}}_{\alpha \beta} H^\alpha H^\beta 
		- \phi \partial_\alpha \Pi^\alpha
		]\\
		&- N^\mu \left[ 
		\epsilon_{\mu \alpha \beta} H^\alpha \Pi^\beta + A_\mu \partial_\alpha \Pi^\alpha
		\right]
	\end{split}
\end{equation}

where $H^\alpha$ is defined in eq. \ref{def_H_alpha}.

If the gravitational fields are very weak, neglecting all but linear terms in these fields yields a good approximation. The parametrisation used for this perturbative approach is

\begin{align}
	\bar{g}^{\alpha \beta} 
	&:= \gamma^{\alpha \beta} + \bar{\varphi}^{\alpha \beta}\\
	\bar{\bar{g}}_{\alpha \beta}
	&:= \gamma_{\alpha \beta} + \bar{\bar{\varphi}}_{\alpha \beta}\\
	{\bar{\bar{\bar{g}}}^\alpha}_\beta
	&:= {\bar{\bar{\bar{\varphi}}}^\alpha}_\beta
	+ \mathcal{O}\left(
	\bar{\bar{\bar{\varphi}}}^2 
	\right)
\end{align}

Substituting this into the Hamiltonian, and neglecting all but the leading, linear order in $\varphi$, yields an approximate Hamiltonian which is valid as long as the gravitational fields are small enough, eg. $\varphi \ll 1$.

Using Jacobi's formula, I obtain

\begin{equation}
	\det{\bar{g}^{\cdot \cdot}} 
	= 1 + \gamma_{\alpha \beta} \bar{\varphi}^{\alpha \beta}
	+ \mathcal{O} \left( \varphi^2 \right)
\end{equation}

and arrive at

\begin{equation} \label{geld_pert_ham}
	\begin{split}
		H = 
		- \left( 1 + A \right) 
		\left[
		\gamma_{\alpha \beta}
		\left( 
		2 \pi \Pi^\alpha \Pi^\beta
		+ \frac{1}{8 \pi} H^\alpha H^\beta 
		\right)
		+ \phi \partial_\alpha \Pi^\alpha
		\right] \\
		- \frac{1}{2} \left( 
		\bar{\bar{\varphi}}^{\alpha \beta} 
		- \bar{\varphi}^{\alpha \beta}
		\right)
		 \left( 
		 \delta^\alpha_\sigma \delta^\beta_\tau 
		- \frac{1}{2} \gamma^{\alpha \beta } \gamma_{\sigma \tau} 
		\right) \\
		\times \left[
		2 \pi 
		\Pi^\alpha \Pi^\beta
		+ \frac{1}{8 \pi} 
		H^\alpha H^\beta 
		\right] \\
		+ \frac{1}{2} \left( 
		\bar{\bar{\varphi}}^{\alpha \beta} 
		+ \bar{\varphi}^{\alpha \beta}
		\right) \\
		\times \left[
		2 \pi 
		\left( 
		\delta^\alpha_\sigma \delta^\beta_\tau 
		- \frac{1}{2} \gamma^{\alpha \beta } \gamma_{\sigma \tau} 
		\right)
		\Pi^\alpha \Pi^\beta
		+ \frac{1}{8 \pi} 
		 \left( 
		- \delta^\alpha_\sigma \delta^\beta_\tau 
		- \frac{1}{2} \gamma^{\alpha \beta } \gamma_{\sigma \tau} 
		\right)
		H^\alpha H^\beta 
		\right] \\
		+ {\bar{\bar{\bar{\varphi}}}^\beta}_\mu 
		\left(
		\delta^{\mu}_\nu \delta^\alpha_\beta 
		- \frac{1}{3}\delta^{\mu}_\beta \delta^\alpha_\nu
		\right) H^\nu \Pi_\alpha \\
		- N^\mu
		\left[
		\epsilon_{\mu \alpha \beta} H^\alpha \Pi^\beta
		+ A_\mu \partial_\alpha \Pi^\alpha
		\right]
	\end{split}
\end{equation}

Confidence about this result arises in comparison with the metric case: Choosing

\begin{equation}
	G^{a b c d} = 
	g^{a c} g^{b d} - g^{a d}g^{b c} 
	- \left(
	-\det{g^{\cdot \cdot}}
	\right) ^{\frac{1}{2}}
	\epsilon^{ a b c d}
\end{equation}

turns birefringent electrodynamics into electrodynamics à la Maxwell - $G$ is then called a metric induced area metric. Using this equation, I determine the value of the area metric gravitational fields in the case where $G$ is metric induced, and obtain

\begin{align}
	\frac{1}{2} 
	\left( 
	\bar{\bar{\varphi}}^{\alpha \beta} 
	- \bar{\varphi}^{\alpha \beta}
	\right) 
	&= \varphi^{\alpha \beta}\\
	\frac{1}{2} 
	\left( 
	\bar{\bar{\varphi}}^{\alpha \beta} 
	+ \bar{\varphi}^{\alpha \beta}
	\right) 
	&= 0 \\
	{\bar{\bar{\bar{\varphi}}}^\alpha}_\beta 
	&= 0
\end{align}

where $\varphi$ is defined in eq. \ref{def_pert_met}. Substituting this into the Hamiltonian \ref{geld_pert_ham} yields the Hamiltonian of Maxwell electrodynamics \ref{mel_pert_ham}, prooving at least partial correctness of the result.

\subsection{Point Particles} \label{sec_pp_am}

The point particle theory that matches birefringent electrodynamics, i.e. that shares the same causality (and, for massless point particles, constitutes the geometric optical limit of birefringent electrodynamics) is again, as in section \ref{sec_pp_met} formulated as soon as the causal structure $P$ of birefringent electrodynamics is inserted into the universal point particle action \ref{act_pp_univ}.

In terms of the area metric $G$ that first appeared in the action \ref{act_gled}, the principle polynomial $P$ which encodes the causal structure reads

\begin{equation}
\begin{split}
	P \left( k \right)
	&= 
	- \frac{4 !}{\left( \epsilon_{e f g h} G^{e f g h }\right)^2}
	\epsilon_{m n p q}
	\epsilon_{r s t u}
	G^{m n r ( a}
	G^{b | p s | c}
	G^{d ) q t u }
	k_a k_b k_c k_d\\
	&=: P^{a b c d} k_a k_b k_c k_d
\end{split}	
\end{equation}

To substitute this into the universal point particle action, I need the components of $P$  in the foliation frame. These have been determined by H. M. Rieser and others, whom I trust to provide correct expressions.

If, to match the accuracy of the  Hamiltonian \ref{geld_pert_ham}, one discards all but the linear oder in the gravitational fields, one obtains

\begin{align}
	P \left( \kappa, \kappa, \kappa, \kappa \right)
	&= 1 - 4A  \label{p0000} \\
	P \left( \tilde{\epsilon}^\alpha, \kappa, \kappa, \kappa \right)
	&= - N^\alpha \\
	P \left( \tilde{\epsilon}^\alpha, \tilde{\epsilon}^\beta, \kappa, \kappa \right)
	&= - \frac{1}{3} \gamma^{\alpha \beta}
	+ \frac{2}{3} A \gamma^{\alpha \beta} \\
	&+ \frac{1}{6} 
	\left[
		- 
		\left( 
			\delta^\alpha_\mu \delta^\beta_\nu 
			+ \gamma^{\alpha \beta} \gamma_{\mu \nu} 
		\right)
		\bar{\varphi}^{\mu \nu}
		+
		\left( 
			\delta^\alpha_\mu \delta^\beta_\nu 
			- \gamma^{\alpha \beta} \gamma_{\mu \nu} 
		\right)
		\bar{\bar{\varphi}}^{\mu \nu}
	\right] \\
	P \left( \tilde{\epsilon}^\alpha, \tilde{\epsilon}^\beta, \tilde{\epsilon}^\gamma, \kappa \right)
	&= N^{( \alpha}\gamma^{\beta \gamma )}\\
	P \left( \tilde{\epsilon}^\alpha, \tilde{\epsilon}^\beta, \tilde{\epsilon}^\gamma, \tilde{\epsilon}^\delta \right)
	&= \gamma^{( \alpha \beta} \gamma^{\gamma \delta )}
	+ 
	\gamma^{( \alpha \beta}
	\left( 
		\delta^\gamma_\mu \delta^{\delta ) }_\nu 
		+ \gamma^{\gamma \delta ) } \gamma_{\mu \nu} 
	\right)
	\bar{\varphi}^{\mu \nu} \\
	&- 
	\gamma^{( \alpha \beta}
	\left( 
		\delta^\gamma_\mu \delta^{\delta ) }_\nu 
		- \gamma^{\gamma \delta ) } \gamma_{\mu \nu} 
	\right)
	\bar{\bar{\varphi}}^{\mu \nu} \label{pabcd}
\end{align}

I read off the zeroth order

\begin{equation}
	P^{a b c d} = \eta^{ ( a b } \eta^{ c d )} + \Sigma^{ a b c d}
\end{equation}

where $\Sigma \sim \mathcal{O} \left( \varphi \right)$.

Using this parametrisation and starting from the universal point particle action, F. P. Schuller obtained

\begin{equation}
	S = 
	m \int \mathrm{d} \tau 
	\left[
		\left( 
			\eta_{a b} \gamma^{\prime a} \gamma^{\prime b} 
		\right)^{\frac{1}{2}}
	 -
	 \frac{1}{4}
	 \frac{
	 \Sigma_{a b c d}
	 \gamma^{\prime a} \gamma^{\prime b} \gamma^{\prime c} \gamma^{\prime d}
	 }{\left( 
			\eta_{a b} \gamma^{\prime a} \gamma^{\prime b} 
		\right)^{\frac{3}{2}}}
		+
		\mathcal{O} \left( \varphi^2 \right)
		+ e \gamma^{\prime a} A_a
	\right]
\end{equation}

where $\gamma^\prime := \frac{\mathrm{d} \gamma \left( \tau \right)}{\mathrm{d} \tau}$. Just as in section \ref{sec_pp_met}, I proceed by parametrising $\gamma$ with the foliation parameter $t$ and  expressing the action in the foliation frame. I yield

\begin{equation}
\begin{split}
	S = 
	m \int \mathrm{d} t 
	[
		\left( 
			1 - v^2
		\right)^{\frac{1}{2}}
	&-
	\frac{1}{4}
	\frac{
	\sum_{n = 0}^4 
	\begin{pmatrix}
		4\\ n
	\end{pmatrix}
	 \Sigma_{\alpha_1 \dots \alpha_n 0 \dots 0}
	 v^{\alpha_1} \dots v^{\alpha_n}
	 }{\left( 
			1 - v^2
		\right)^{\frac{3}{2}}} \\
		&+ e \left(
			\phi + A \phi +  A_\alpha \left( N^\alpha + v^\alpha\right)
		\right) 
	]
	+
	\mathcal{O} \left( \varphi^2 \right)
\end{split}
\end{equation}

where again $v = \frac{\mathrm{d} \lambda (t)}{\mathrm{d} t } =: \dot{\lambda} (t) $ and $v^2 := \gamma_{\alpha \beta} v^\alpha v^\beta $. 

I proceed just as in section \ref{sec_pp_met}, and find that I obtain the same results concerning the momenta and velocities to order $\mathcal{O} \left( 1
	\right)$ as given in eq. \ref{momentum_pp_met} - \ref{veloc_pp_met}. Again, this knowledge suffices to derive the Hamiltonian to order  $\mathcal{O} \left( \varphi \right)$ through the Legendre transform.
	
	After inserting the expressions \ref{p0000} - \ref{pabcd}, the approximate Hamiltonian for a charged point mass in an electromagnetic field that obeys the causal structure of birefringent electrodynamics reads
	
\begin{equation}
	\begin{split}
		H = 
		&- \left( 1 + A \right) \left( E_k + e \phi \right) 
		- N^\alpha p_\alpha \\
		&- \frac{1}{2 E_k} \left[ 
			\frac{1}{2} 
			\left( 
				\bar{\varphi}^{\alpha \beta} 
				- \bar{\bar{\varphi}}^{\alpha \beta}
			\right) 
			k_\alpha k_\beta
			+
			\frac{1}{2} 
			\left( 
				\bar{\varphi}^{\alpha \beta} 
				+ \bar{\bar{\varphi}}^{\alpha \beta}
			\right) 
			\gamma_{\alpha \beta} \gamma^{\mu \nu}
			k_\mu k_\nu
		\right]
	\end{split}
\end{equation}

with the same notational conventions as in eq. \ref{ham_pp_met}

\subsection{Gravitational Waves}

Collecting the results from sec. \ref{sec_em_am} and \ref{sec_pp_am}, I am now in possession of complete matter Hamiltonian:

\begin{equation}
	H_{\text{M}}  = \sum_i H^{(i)}_{\text{Point particle}} + H_{\text{Electromagnetism}}
\end{equation}


 This is necessary if one wishes to calculate the gravitational waves generated by a matter system, because derivatives of the matter Hamiltonian appear as imhomogenities or source terms in the equations of motion of the gravitational fields. To determine the gravitational radiation emitted from a specific system, these source terms must be evaluated on a specific solution of the matter equations of motion - I will evaluate them on the solution presented in sec. \ref{sec_mot_bin}.

Now, firstly, I present the linearised equations of motion for the gravitational fields that form the causal structure of birefringent electrodynamics, as obtained by the members of the constructive gravity group. These equations are published in an article called "Gravitational closure of weakly birefringent electrodynamics". In this work, I restrict myself to the special case called the $\xi = 0$ case. In this special case, some parts of the full theory are switched of (by setting the respective constants of nature to zero).

The linearised equations of motion approximate the full nonlinear ones, solutions of the former accurately resemble solutions of the later as long as the field strengths are very small, i.e. as $\varphi \ll 1$.

The fields that obey the equations of motion stem from a decomposition of the fields $\bar{\varphi}$, $\bar{\bar{\varphi}}$ and $\bar{\bar{\bar{\varphi}}}$ into scalar fields, divergence free vector fields and traceless, divergence free  symmetric tensor fields :

\begin{align}
	\bar{\varphi}_{\alpha \beta} 
	&=: 
	\tilde{F} \gamma_{\alpha \beta} 
	+ \Delta_{\alpha \beta} F 
	+  2 \partial_{ ( \alpha} F_{\beta )} 
	+  F_{\alpha \beta} \label{def_F}  \\
	\bar{\bar{\varphi}}_{\alpha \beta} 
	&=:
	\tilde{E} \gamma_{\alpha \beta} 
	+ \Delta_{\alpha \beta} E 
	+  2 \partial_{ ( \alpha} E_{\beta )} 
	+  E_{\alpha \beta} \label{def_E} \\
	\bar{\bar{\varphi}}_{\alpha \beta} 
	&=: 
	\Delta_{\alpha \beta} C 
	+  2 \partial_{ ( \alpha} C_{\beta )} 
	+  C_{\alpha \beta} \label{def_C}
\end{align}

I shall now give my version of the equations of motion for these fields. It  differs from the version presented in the above quoted article in seven aspects:

\begin{enumerate}
	\item As already stated, I focus on the $\xi = 0$ case. This becomes manifest in some of the $\kappa$-constants being set to zero.
	\item For some historic reason, the authors of the above quoted article chose the gauge-fixing $E^\alpha = F = B = 0$. I personally find that gauge fixing rather impractical, and prefer to use $B^\alpha = B = F - E = 0 $ instead.
	\item It appears to me that for both practical and conceptual reasons, instead of working with the fields $F_A$, $E_A$ and $C_A$ (where the $A$ stands for any index-type, I am talking about scalars, vectors and tensors here) one should rather work with the fields $ V_A := F_A - E_A$,  $U_A = E_A + F_A$ and $I_A = 2 C_A$. The practical reasons for this redefinition of variables reveal themselves to everyone who can be bothered to follow my calculations, whereas the conceptual reason is that $V_A$ captures exactly the metric degrees of freedom, and is therefore the only field which would survive the metric limit described at the end of sec. \ref{sec_em_am}.
	\item Since it reduces redundancy in notation, I partially reverse the decomposition presented in eq. \ref{def_F} - \ref{def_C}, and work with the trace free tensor fields 

	\begin{align}
		\bar{U}_{\alpha \beta} &= \Delta_{\alpha \beta} U
		+  2 \partial_{ ( \alpha} U_{\beta )} 
		+  U_{\alpha \beta}\\
		\bar{I}_{\alpha \beta} &= \Delta_{\alpha \beta} I
		+  2 \partial_{ ( \alpha} I_{\beta )} 
		+  I_{\alpha \beta}
	\end{align}
	
		instead of with the parts from which they are assembled.
		
	\item In his analysis of the linearised equations of motion in vacua and their solutions, N. Alex found that certain combinations of constants must vanish in order to make flat Minkowski spacetime a stable solution of the theory. Since this criterion is non-negotiable, I impose these stability conditions on the respective constants. The conditions are given in detail in sec. \ref{sec_appen}.
	\item Following N. Alex in another aspect, I refuse to work with the $\kappa$-constants given in the above quoted article. Instead, I shall use constants labeled with $m$, $s$, $v$ and $t$. In sec. \ref{sec_appen}, I provide the equations to convert the former into the latter.
	\item I have combined the equation in a way as to clearly separate truly dynamical equations, i.e. wave equations, from the constraint equations. The way the equations are written down allows to systematically solve them in few steps.
\end{enumerate}

\begin{landscape}

Wave equations:

\begin{align}
	-m_4 
	\left[ 
		\frac{
			\delta H_{\text{M}}
		}{
			\delta \bar{\varphi}^{\alpha \beta}
		}
		+
		\frac{
			\delta H_{\text{M}}
		}{
			\delta \bar{\bar{\varphi}}^{\alpha \beta}
		}
	\right]^{\text{TF}} 
	+ m_2 
	\left[ 
		\frac{
			\delta H_{\text{M}}
		}{
			\delta \bar{\bar{\bar{\varphi}}}^{\alpha \beta}
		}
	\right]^{\text{TF}} 
	&=
	\left( 
		m_3 m_4 - m_1 m_2 
	\right)
	\Box \bar{I}_{\alpha \beta}
	+ 
	\left( 
		m_4^2 - m_2^2 
	\right)
	\bar I_{\alpha \beta}
	\label{w_eq_I}
	\\
	-m_4 
	\left[ 
		\frac{
			\delta H_{\text{M}}
		}{
			\delta \bar{\bar{\bar{\varphi}}}^{\alpha \beta}
		}
	\right]^{\text{TF}}
	+ m_2
	\left[ 
		\frac{
			\delta H_{\text{M}}
		}{
			\delta \bar{\varphi}^{\alpha \beta}
		}
		+
		\frac{
			\delta H_{\text{M}}
		}{
			\delta \bar{\bar{\varphi}}^{\alpha \beta}
		}
	\right]^{\text{TF}}   
	&=
	\left( 
		m_3 m_4 - m_1 m_2 
	\right)
	\Box \bar{U}_{\alpha \beta}
	+ 
	\left( 
		m_4^2 - m_2^2 
	\right)
	\bar U_{\alpha \beta}
	\label{w_eq_U}
	\\
	-
	\left[ 
		\frac{
			\delta H_{\text{M}}
		}{
			\delta \bar{\varphi}^{\alpha \beta}
		}
		-
		\frac{
			\delta H_{\text{M}}
		}{
			\delta \bar{\bar{\varphi}}^{\alpha \beta}
		}
	\right]^{\text{TT}} 
	&=
	\left( 
		t_1 - t_3
	\right)
	\Box \bar{V}_{\alpha \beta}
	\label{w_eq_V}
\end{align}

\begin{multline}
	\left(
		3 \left( s_1 + s_2\right) + 2 s_8
	\right)
	\gamma^{\alpha \beta}
	\left[ 
		\frac{
			\delta H_{\text{M}}
		}{
			\delta \bar{\varphi}^{\alpha \beta}
		}
		-
		\frac{
			\delta H_{\text{M}}
		}{
			\delta \bar{\bar{\varphi}}^{\alpha \beta}
		}
	\right]
	-
	3 \left(s_1 + s_2 \right) 
	\gamma^{\alpha \beta}
	\left[ 
		\frac{
			\delta H_{\text{M}}
		}{
			\delta \bar{\varphi}^{\alpha \beta}
		}
		+
		\frac{
			\delta H_{\text{M}}
		}{
			\delta \bar{\bar{\varphi}}^{\alpha \beta}
		}
	\right]
	+
	\left(
		3 \left(s_1 + s_2 \right) - s_8
	\right)
	\frac{\delta H_{\text{M}}}{\delta N}
	\\
	=
	- \left(
	108 \left(s_1 + s_2 \right)^2
	+ 72 \left(s_1 + s_2 \right) s_8
	- 18 \left(s_1 + s_2)\right) s_{28}
	+ 12 s_8^2
	\right)
	\Box \tilde{U}
	+ 
	18 \left( s_1 + s_2\right) s_{32} \tilde{U}
	\label{w_eq_Us}
\end{multline}
	
Constraint equtions:

\begin{align}
	\frac{\delta H_{\text{M}}}{\delta N}
	+
	\left( 
		6 \left( s_1 + s_2 \right) + 4 s_8
	\right)
	\Delta \tilde{U}
	&=
	2 \left( s_1 + s_2 \right) \Delta^2 V
	\label{c_eq_V}
	\\
	- \frac{\delta H_{\text{M}}}{\delta N}
	- 2 \gamma^{\alpha \beta}
	\left[ 
		\frac{
			\delta H_{\text{M}}
		}{
			\delta \bar{\varphi}^{\alpha \beta}
		}
		-
		\frac{
			\delta H_{\text{M}}
		}{
			\delta \bar{\bar{\varphi}}^{\alpha \beta}
		}
	\right]
	-
	\left(
	18 \left( s_1  + s_2 \right) +  12 s_8
	\right) \ddot{\tilde{U}}
	-
	\left(
	12 \left( s_1  + s_2 \right) - 4 s_8 
	\right)
	\Delta \tilde{U}
	&=
	24 \left( s_1 + s_2 \right) \Delta A
	\label{c_eq_A}
	\\
	-
	\left[
	\frac{\delta H}{\delta N^\alpha}
	\right]^{\text{V}}
	&=
	2 \left( v_1 + v_2 \right) \Delta \dot{V}_\alpha
	\label{c_eq_Vv}
\end{align}
	
\end{landscape}

A trace free symmetric (2,0) tensor field on a three dimensional manifold yields 5 degrees of freedom, a divergence free vector yields two degrees of freedom, as does a transverse (i.e. divergence free) trace free symmetric (2,0) tensor field. A scalar field yields one degree of freedom.

Thus, the wave eq. \ref{w_eq_I} - \ref{w_eq_Us} determine 13 degrees of freedom, which shall be referred to as propagating degreed of freedom. The constraint eq. \ref{c_eq_V} - \ref{c_eq_Vv} determine 4 degrees of freedom, and further 4 degreed of freedom are determined by the gauge fixing $N^\alpha = \tilde{V} = 0$. 

I shall now show how the sourced wave equations can be solved, and hence determine the gravitational radiation emitted by the orbiting charges discussed in \ref{sec_mot_bin}.

Consider the problem of solving the differential equation

\begin{equation} \label{massive_kg}
	\left( \Box + M^2 \right) X_A \left(x, t \right) 
	=
	\rho_A \left(x, t \right)  
\end{equation}

which I refer to as massive Klein Gordon equation. I shall be discussing the solution for this equation in the case that $M^2 \ge 0$.

After Fourier transforming the time dependence of both $X_A$ and $\rho$, eq. \ref{massive_kg} implies

\begin{equation} \label{massive_kg_fourier}
	\left( - \Delta - \omega^2 + M^2  \right) X_A \left( x, \omega \right)
	=
	\rho_A \left(x, \omega \right)
\end{equation}

This is the so called screened Poisson equation, which I tackle with the method of Green's functions. The defining equation for the Green function is 

\begin{equation}
	\left( c + \Delta \right) G_x \left(y  \right) = \delta_y \left( y \right)
\end{equation}

The solutions (should I present the derivation here?) are

\begin{equation}
	G_x \left( y \right) 
	=
	- \frac{1}{4 \pi \left| x - y\right|} 
	e^{- \sqrt{\left| c \right|}\left| x - y\right|}
\end{equation}

for $c < 0$ and

\begin{equation}
	G^\pm_x \left( y \right) 
	=
	- \frac{1}{4 \pi \left| x - y\right|} 
	e^{\pm i \sqrt{\left| c \right|}\left| x - y\right|}
\end{equation}

for $c > 0$, where there are two solutions. These correspond to absorption and radiation respectively, as is discussed in any standard textbook about field theory. The sign I need to pick to decide for the radiation related one is $- \text{sgn} \left( \omega \right)$. Further, $\left|x \right| := \gamma_{\alpha \beta}x^\alpha x^\beta $.

According to the method of Green's functions, the solution of eq. \ref{massive_kg_fourier} is 

\begin{equation} \label{massive_kg_sol}
	\begin{split}
		X_A \left( x, \omega \right) 
		&= 
		- \int \mathrm{d} y^3 G_x \left( y \right) \rho_A \left( y, \omega \right)
		\\ 
		&= 
		\int \mathrm{d} y^3 \frac{ \rho_A \left( y, \omega \right)}
		{4 \pi \left| x - y \right|}
		e^{-i \omega \sqrt{1 - M^2 / \omega^2} \left| x - y \right|}
	\end{split}
\end{equation} 

in the case $\omega ^2 > M^2 $, where I chose the radiation related Green function.

So far, the solution is exact, but somewhat impractical. I shall therefore present a few steps towards approximating the  above solution, which is much easier, but less general. Namely, the approximation I am about to present is very accurate only if 

\begin{enumerate}
	\item the source is compact
	\item the source moves slowly, i.e. typical velocities of the matter that makes up the source are much smaller than the speed of light
	\item the wave is observed far away from the source.
	
\end{enumerate}

Without loosing generality, I assume that the source is located at the origin of the coordinate system $\{ y^\alpha \}$, and, as it is compact, is contained in a sphere of diameter $d$ around the origin. Further, I evaluate the field $X_A$ at a position $x$ such that $\left| x \right| \gg d$. 

The integrand in \ref{massive_kg_sol} contains $\rho_A$, which is only supported within the sphere of diameter $d$ around the origin. The integrand thus vanishes unless $\left| y \right| < d $,  or equivalently $ \epsilon := \left| y \right| / \left| x \right| \ll 1$. Expanding \ref{massive_kg_sol} in $\epsilon$ and keeping only the dominant contribution yields

\begin{multline}
	X_A \left( x, \omega \right) 
	=
	\frac{e^{-i \omega \sqrt{1 - M^2 / \omega^2} r}}{4 \pi r}
	\\
	\int \mathrm{d} y^3  \rho_A \left( y, \omega \right)
		\exp 
		\left( 
			-i \omega \sqrt{1 - M^2 / \omega^2} \hat x^\alpha  y_\alpha 
		\right)
		\times
		\left( 
			1 + \mathcal{O} \left( \epsilon \right)
		\right)
\end{multline}

where $r := \left| x \right| $ and $\hat x := x / r$.

Finally, note that the order of magnitude of a typical velocity within the source is $v \sim \omega d$, so according to the requirement I assume $\omega d \ll 1$, which allows a second expansion. I arrive at

\begin{multline} \label{gen_sol}
	X_A \left( x, \omega \right)
	 = \frac{e^{- i \omega r\sqrt{1 - M^2 / \omega^2}}}{4 \pi r }
	\sum_{n = 0}^\infty
	\frac{1}{n!} 
	\left( 
		i \omega \sqrt{1 - M^2 / \omega^2}
	\right)^n
	\\
	\hat x^{\alpha_1} \dots \hat x^{\alpha_n}
	\int \mathrm{d }y^3 \, y_{\alpha_1}  \dots y_{\alpha_n}\,\,
	\rho_A \left( y, \omega \right)
	\times
		\left( 
			1 + \mathcal{O} \left( \epsilon \right)
		\right)
\end{multline}

Inspecting this expression, one notices that the source is now organised into a multipole expansion, where higher moments are suppressed by higher powers of $v$. To obtain a consistent result, it is necessary to keep in mind that also $\rho_A$ contains $v$, and thus must be expanded, too. Then, all contributions up to some specified order of $v$ must be collected.

With this general solution at hand, I will now derive the solutions of the wave equations \ref{w_eq_I} - \ref{w_eq_Us} to an accuracy of $\mathcal{O} \left( v \right)$, taking the source matter to be the system composed of two orbiting charges and an electric field, which I analysed in \ref{sec_mot_bin}. 

As for eq. \ref{w_eq_I}: The source term for the field denoted by $\bar I_{\alpha \beta}$ is

\begin{equation}
	\left( \rho_{\bar I} \right)_{\alpha \beta}
	= -\sigma_{1}
	\mathcal{P}_{\alpha  \beta}^{\mu \nu}
	\left[ 
		\frac{
			\delta H_{\text{M}}
		}{
			\delta \bar{\varphi}^{\mu \nu}
		}
		+
		\frac{
			\delta H_{\text{M}}
		}{
			\delta \bar{\bar{\varphi}}^{\mu \nu}
		}
	\right]
	+ \sigma_2 
	\mathcal{P}_{\alpha  \beta}^{\mu \nu}
	\left[ 
		\frac{
			\delta H_{\text{M}}
		}{
			\delta \bar{\bar{\bar{\varphi}}}^{\mu \nu}
		}
	\right]
\end{equation}


where $\mathcal{P}_{\alpha  \beta}^{\mu \nu} = \delta^\mu_\alpha \delta^\nu_\beta - \frac{1}{3} \gamma_{\alpha \beta}^{\mu \nu}$ and

\begin{align}
	\sigma_1 &= \frac{m_4}{ m_4 m_3  - m_1 m_2} 
	&
	\sigma_2 &= \frac{m_2}{ m_4 m_3  - m_1 m_2}
\end{align}

I evaluate the source term for the orbiting charges to order $v^2$ and obtain

\begin{multline}
		\int \mathrm{d}^3 y \left( \rho_{\bar I} \right)_{\alpha \beta}
		\left( y, \omega \right)\\
		=
		- \int \mathrm{d} t \,e^{-i \omega t}
		\int \mathrm{d}^3 y\,
		\frac{\sigma_1}{8 \pi} \mathcal{P}^{\mu \nu}_{\alpha \beta}
		\left( 
		E_\mu E_\nu - \frac{1}{2} \gamma_{\mu \nu} E^\sigma E_\sigma
		\right)\\
		=
		- \frac{\sigma_1}{8 \pi} \mathcal{P}^{\mu \nu}_{\alpha \beta}
		\int \mathrm{d} t \,e^{-i \omega t}
		\int \mathrm{d}^3 y\,
		\partial^\tau y_\mu
		\left( 
		E_\tau E_\nu - \frac{1}{2} \gamma_{\mu \nu} E^\sigma E_\sigma 
		\right)\\
		=
		\frac{\sigma_1}{8 \pi} \mathcal{P}^{\mu \nu}_{\alpha \beta}
		\int \mathrm{d} t \,e^{-i \omega t}
		\int \mathrm{d}^3 y\,
		y_\mu
		\left(
		E_\nu \partial_\tau E^\tau
		+
		E^\tau \left[
		\partial_\tau E_\nu - \partial_\nu E_\tau
		\right]
		\right)
\end{multline}

using integration by parts. Eq. \ref{eom_em_2} combined with the result in eq. \ref{H_bin} grants that $ \partial_\tau E_\nu - \partial_\nu E_\tau = 0$, while, via eq. \ref{eom_em_3} and the result \ref{E_bin} , I arrive at

\begin{multline}
	\int \mathrm{d}^3 y \left( \rho_{\bar I} \right)_{\alpha \beta}
	\left( y, \omega \right)
	=
	-\frac{\pi \mu \sigma_1 d^2 \omega^2_{\text{bin}}}{2} \times
	\\
	\left[
	\frac{\delta\left( \omega \right)}{3}
	\begin{pmatrix}
	1 & & \\ & 1 & \\ & & -2
	\end{pmatrix}_{\mu \nu}
	\right.
	+
	\frac{\delta \left( \omega - 2 \omega_{\text{bin}} \right)}{2}
	\begin{pmatrix}
	1 & -i & \\ -i & -1 & \\ & & &
	\end{pmatrix}_{\mu \nu}
	\\
	\left.
	+
	\frac{\delta \left( \omega + 2 \omega_{\text{bin}} \right)}{2}
	\begin{pmatrix}
	1 & i & \\ i & -1 & \\ & & &
	\end{pmatrix}_{\mu \nu}
	\right] + \text{infinite contributions}
\end{multline}


In the course of this calculation, I obtained 

\begin{multline} \label{usefull_result}
	\frac{1}{8 \pi} \int \mathrm{d}^3 y\,
		\left( 
		E_\mu E_\nu - \frac{1}{2} \gamma_{\mu \nu} E^\sigma E_\sigma
		\right)\\
		=
		\frac{\mu d^2 \omega_{\text{bin}}^2}{2} 
		\begin{pmatrix}
			\cos \omega_{\text{bin}} t \\
			\sin \omega_{\text{bin}} t \\
			\\
		\end{pmatrix}_{\nu}
		\begin{pmatrix}
			\cos \omega_{\text{bin}} t \\
			\sin \omega_{\text{bin}} t \\
			\\
		\end{pmatrix}_{\mu}
		+ 
		\text{infinite contributions}
\end{multline}


as an intermediate result. I also used eq. \ref{freq_bin} to derive this result.
The infinite contributions arise as the electric field strength must be evaluated at its poles - since I do not know how to make sense of this, I discard these contributions from here onwards. I insert that into the general solution \ref{gen_sol} and transform the resulting amplitude into the time domain to obtain the final expression for the gravitational wave:

\begin{multline} \label{sol_I}
	\bar I_{\alpha \beta} \left( x, t\right)
	=
	\\
	- \frac{\sigma_1}{4 \pi r}
	\frac{\mu d^2 \omega_{\text{bin}}^2}{4} 
	\begin{pmatrix}
		\cos{2 \omega_{\text{bin}} \tilde t}
		 & \sin{2 \omega_{\text{bin}} \tilde t}
		 &
		 \\
		 \sin{2 \omega_{\text{bin}} \tilde t}
		 & - \cos{2 \omega_{\text{bin}} \tilde t}
		 &
		 \\
		 &
		 &
		 &
	\end{pmatrix}_{\mu \nu}
	+ \text{cons.}
	+ \mathcal{O} \left( \omega^3 \right)
\end{multline}

where $\tilde t = t - r \sqrt{1 - M^2_{\hat I} / \left( 2 \omega_{\text{bin}}\right)^2 }$ and $M_{\hat I }^2  =  \left( m_4^2 - m_2^2\right) / \left( m_3 m_4 - m_1 m_2 \right) $. After a short inspection of the similarities between eq. \ref{w_eq_I} and eq. \ref{w_eq_U}, one will agree with me that the solution of \ref{w_eq_U} can be easily obtained from \ref{sol_I}:

\begin{equation}
	\bar U _{\alpha \beta} = - \frac{\sigma_2}{\sigma_1} \bar I_{\alpha \beta}
\end{equation}

Next, I turn to the solution of eq. \ref{w_eq_Us}. The relevant contributions to the source term, evaluated on the particular solution presented in sec. \ref{sec_mot_bin}, are

\begin{align}
	\gamma^{\alpha \beta} 
	\left[ 
		\frac{
			\delta H_{\text{M}}
		}{
			\delta \bar{\varphi}^{\alpha \beta}
		}
		-
		\frac{
			\delta H_{\text{M}}
		}{
			\delta \bar{\bar{\varphi}}^{\alpha \beta}
		}
	\right]
	&=
	-\frac{1}{2} \frac{k_1^2}{E_{k_1}} \delta_{\lambda_1} 
	-\frac{1}{2} \frac{k_2^2}{E_{k_2}} \delta_{\lambda_2} 
	- \frac{1}{16 \pi} E^2 \\
		\gamma^{\alpha \beta} 
	\left[ 
		\frac{
			\delta H_{\text{M}}
		}{
			\delta \bar{\varphi}^{\alpha \beta}
		}
		+
		\frac{
			\delta H_{\text{M}}
		}{
			\delta \bar{\bar{\varphi}}^{\alpha \beta}
		}
	\right]
	&=
	-\frac{3}{2} \frac{k_1^2}{E_{k_1}} \delta_{\lambda_1} 
	-\frac{3}{2} \frac{k_2^2}{E_{k_2}} \delta_{\lambda_2} 
	- \frac{1}{16 \pi} E^2 \\
	\frac{\delta H }{\delta N} 
	&= 
	- E_{k_1} \delta_{\lambda_1} 
	-E_{k_2} \delta_{\lambda_2} 
	- \frac{1}{8 \pi} E^2 
\end{align}

Combined, I obtain the source term

\begin{equation}
	\rho_{\tilde U} = 
	- \frac{3 \left( s_1 + s_2\right) - s_8 }{\kappa}
	\left[
	\frac{m_1^2}{E_{k_1}} \delta_{\lambda_1}
	+
	\frac{m_2^2}{E_{k_2}} \delta_{\lambda_2}
	\right]
	-\frac{3 \left( s_1 + s_2 \right)}{\kappa} \frac{E^2}{8 \pi}
\end{equation}

where, for brevity, I defined

\begin{equation}
	\kappa = 108 \left(s_1 + s_2 \right)^2
	+ 72 \left(s_1 + s_2 \right) s_8
	- 18 \left(s_1 + s_2)\right) s_{28}
	+ 12 s_8^2
\end{equation}

Next, I must analyse the different moments of that source. The 0th moment, or monopole, does not vanish, but is constant:

\begin{equation}
	\int \mathrm{d} y^3 \rho_{\tilde U} \left( y, \omega \right) 
	=
	\delta \left( \omega \right)
\end{equation}

The monopole does therefore contribute only to a static configuration, but not to gravitational radiation.

The 1st moment, or dipole, vanishes to leading order due to the symmetry of the source:

\begin{equation}
	\int \mathrm{d} y^3 \rho_{\tilde U} \left( y, \omega \right) y^\alpha
	=
	0 + \mathcal{O} \left( v^2  \right)
\end{equation} 

Since the dipole moment enters the Amplitude of the wave with an additional factor $\omega$ (see eq. \ref{gen_sol}) , it does not contribute at order $v^2$.

The second moment, or quadropole, picks up a factor of $\omega^2$ when entering the amplitude of the wave, and therefore contributes to overall order $v^2$ only through its 0th order. I use 

\begin{align}
	\rho_{\tilde U} &= 
	- \frac{3 \left( s_1 + s_2\right) - s_8 }{\kappa}
	\left[
	\frac{m_1^2}{E_{k_1}} \delta_{\lambda_1}
	+
	\frac{m_2^2}{E_{k_2}} \delta_{\lambda_2}
	\right]
	-\frac{3 \left( s_1 + s_2 \right)}{\kappa} \frac{E^2}{8 \pi}
	\\
	&= 
	- \tau
	\left[
	m_1 \delta_{\lambda_1}
	+
	m_2 \delta_{\lambda_2}
	\right]
	+
	\mathcal{O} \left( v^2  \right)
\end{align}

where $\tau := \left( 3 \left( s_1 + s_2\right) - s_8 \right) / \kappa$, and 
obtian:

\begin{multline}
	\int \mathrm{d}y^3 \rho_{\tilde U} \left( y , \omega \right) 
	y_\alpha y_\beta
	=
	- \pi \mu \tau d^2 \times
	\\
	\left[
	\delta\left( \omega \right)
	\begin{pmatrix}
	1 & & \\ & 1 & \\ & &
	\end{pmatrix}_{\alpha \beta}
	\right.
	+
	\frac{\delta \left( \omega - 2 \omega_{\text{bin}} \right)}{2}
	\begin{pmatrix}
	1 & -i & \\ -i & -1 & \\ & & &
	\end{pmatrix}_{\alpha \beta}
	\\
	\left.
	+
	\frac{\delta \left( \omega + 2 \omega_{\text{bin}} \right)}{2}
	\begin{pmatrix}
	1 & i & \\ i & -1 & \\ & & &
	\end{pmatrix}_{\alpha \beta}
	\right]
\end{multline}

Again, inserting this into eq. \ref{gen_sol} and transforming back into time domain yields the final wave:

\begin{multline} \label{sol_Us}
	\tilde U \left( x, t\right)
	=
	\\
	\frac{\tau \mu d^2 \omega_{\text{bin}}^2}{4 \pi r}
	\left(
		1 - \frac{M^2_{\tilde U} }{ \left( 2 \omega_{\text{bin}}\right)^2}
	\right)
	\hat x^\alpha 
	\hat x^\beta
	\begin{pmatrix}
		\cos{2 \omega_{\text{bin}} \tilde t}
		 & \sin{2 \omega_{\text{bin}} \tilde t}
		 &
		 \\
		 \sin{2 \omega_{\text{bin}} \tilde t}
		 & - \cos{2 \omega_{\text{bin}} \tilde t}
		 &
		 \\
		 &
		 &
		 &
	\end{pmatrix}_{\alpha \beta}
	\\
	+ \text{const.}
	+ \mathcal{O} \left( \omega^3 \right)
\end{multline}

where $\tilde t$ is defined as above and 

\begin{equation}
	M_{\tilde U}^2 := - \frac{18 \left( s_1 + s_2 \right) s_{32} }{\kappa}
\end{equation}

The last equation to solve is \ref{w_eq_V}. This equation differs from the other equations in two aspects:

\begin{enumerate}
	\item It is a massless wave equation
	\item Both the field that fulfils the equation and the source term are traceless and transverse.
\end{enumerate}

The first aspect is easily dealt with by just simply setting $M^2$ to zero in eq. \ref{gen_sol}. The second aspect poses a problem, since even if both sides of the eq. \ref{massive_kg} are transverse, the solution in \ref{gen_sol} is not. I fact, the exact solution \ref{massive_kg_sol} is transverse if the source is transverse. In the following approximations, this property gets lost. I chose the treatment presented in most standard texts about gravitational waves: I solve the equation for the traceless part of the source and  finally take the transverse part of that solution. N. Alex provided insight in this problem, and might be able to solve it completely.

The relevant source term is

\begin{multline}
	\left( \rho_V \right)_{\alpha \beta}
	= - \frac{1}{t_1 - t_3}
	\mathcal{P}_{\alpha \beta}^{\mu \nu}
	\left[ 
		\frac{
			\delta H_{\text{M}}
		}{
			\delta \bar{\varphi}^{\mu \nu}
		}
		-
		\frac{
			\delta H_{\text{M}}
		}{
			\delta \bar{\bar{\varphi}}^{\mu \nu}
		}
	\right]\\
	=
	\vartheta
	\mathcal{P}_{\alpha \beta}^{\mu \nu} 
	\left[
	\frac{1}{2} \frac{k_{1 \alpha} k_{1 \beta}}{E_{k_1}} \delta_{\lambda_1} 
	+
	\frac{1}{2} \frac{k_{2 \alpha} k_{2 \beta}}{E_{k_2}} \delta_{\lambda_2} 
	\right.
	\\
	\left.
	- \frac{1}{8 \pi} 
	\left( \delta_\alpha^\sigma \delta^\tau_\beta - \frac{1}{2} \gamma_{\alpha \beta} \gamma^{\sigma \tau} \right) 
	E_{\sigma} E_{\tau}
	\right]
\end{multline}

where $\vartheta := 1 / \left( t_1 - t_3 \right)$. Now using the results from sec. \ref{sec_mot_bin} and eq. \ref{usefull_result}, I determine the leading order contribution to the amplitude to be

\begin{multline}
	\int \mathrm{d} y^3 \left( \rho_V \right)_{\alpha \beta}
	\left( 
	y, \omega
	\right)
	=
	- \frac{\vartheta \mu \omega_{\text{bin}}^2 d^2 \pi }{2}
	\times
	\\
	\left[
	\frac{\delta \left( \omega - 2 \omega_{\text{bin}} \right)}{2}
	\begin{pmatrix}
	1 & -i & \\ -i & -1 & \\ & & &
	\end{pmatrix}_{\alpha \beta}
	+
	\frac{\delta \left( \omega + 2 \omega_{\text{bin}} \right)}{2}
	\begin{pmatrix}
	1 & i & \\ i & -1 & \\ & & &
	\end{pmatrix}_{\alpha \beta}
	\right]
\end{multline}

I arrive at the tracefree solution

\begin{multline} \label{sol_V_TF}
	V^{\text{TF}} \left( x, t\right)_{\alpha \beta}
	=
	\\
	\frac{\varphi \mu d^2 \omega_{\text{bin}}^2}{8 \pi r}
	\begin{pmatrix}
		\cos{2 \omega_{\text{bin}} \tilde{ \tilde{ t}}}
		 & \sin{2 \omega_{\text{bin}} \tilde{ \tilde{ t}}}
		 &
		 \\
		 \sin{2 \omega_{\text{bin}} \tilde{ \tilde{ t}}}
		 & - \cos{2 \omega_{\text{bin}} \tilde{ \tilde{ t}}}
		 &
		 \\
		 &
		 &
		 &
	\end{pmatrix}_{\alpha \beta}
	\\
	+ \mathcal{O} \left( \omega^3 \right)
\end{multline}

where $\tilde{ \tilde{ t}} := t - r$. The last step consists in applying the transverse projector

\begin{equation}
	\mathcal{P}_\mu^\alpha 
	:= \delta_\mu^\alpha - \left( \Delta^{-1} \right) 
	\partial^\alpha \partial_\mu
\end{equation}

to both indices of $V^{\text{TF}}_{\alpha \beta}$. Note that if used on expression \ref{sol_V_TF}, a partial derivative can be substituted by

\begin{equation}
	\partial^\alpha \partial_\beta 
	= 
	- 4 \omega_{\text{bin}}^2 
	\frac{x^\alpha x_\beta}{r^2} + \mathcal{O} \left( \frac{1}{r} \right)
\end{equation}

Therefore, the projector becomes

\begin{equation}
	\mathcal{P}_\mu^\alpha 
	= \delta_\mu^\alpha 
	+ \frac{x^\alpha x_\mu}{r^2}
	+ \mathcal{O} \left( \frac{1}{r} \right)
	= \delta_\mu^\alpha 
	+ \hat x^\alpha \hat x_\mu
	+ \mathcal{O} \left( \frac{1}{r} \right)
\end{equation}

and I find the final result

\begin{multline} \label{sol_V_TT}
	V
	\left( x, t\right)_{\mu \nu}
	=
	\frac{\varphi \mu d^2 \omega_{\text{bin}}^2}{8 \pi r}
	\\
	\left(
		\delta_\mu^\alpha 
		+ \hat x^\alpha \hat x_\mu
	\right)
	\left(
		\delta_\nu^\beta
		+ \hat x^\beta \hat x_\nu
	\right)
	\begin{pmatrix}
		\cos{2 \omega_{\text{bin}} \tilde{ \tilde{ t}}}
		 & \sin{2 \omega_{\text{bin}} \tilde{ \tilde{ t}}}
		 &
		 \\
		 \sin{2 \omega_{\text{bin}} \tilde{ \tilde{ t}}}
		 & - \cos{2 \omega_{\text{bin}} \tilde{ \tilde{ t}}}
		 &
		 \\
		 &
		 &
		 &
	\end{pmatrix}_{\alpha \beta}
	\\
	+ \mathcal{O} \left( \omega^3 \right)
\end{multline}


\section{Conventions}

\begin{itemize}
	\item latin indices are spacetime indices
	\item greek indices are screen manifold indices
\end{itemize}

\begin{align} \label{conv_eps}
	\epsilon^{0 1 2 3} &=  - 1\\
	\Box &= \left( \partial / \partial t \right)^2 -  \Delta
\end{align}

Fourier transform:

\begin{align}
	f \left( \omega \right) 
	&:= \int \mathrm{d} t \, e^{-i \omega t} f \left( t \right) \\
	f \left( t \right)
	&= \int \frac{\mathrm{d} \omega}{2 \pi} e^{i \omega t}  f \left( \omega \right)\\
	\Rightarrow \delta \left( \omega_1 - \omega_2 \right)
	&=  \int \frac{\mathrm{d} t}{2 \pi}  
	e^{i \left( \omega_1 - \omega_2 \right) t}
\end{align}

\section{Appendix} \label{sec_appen}

\subsection{Stability conditions}

\begin{align}
	\left(t_1 + t_3 \right) t_13 &= 2 t_5 t_11\\
	- v_2 v_9 &= v_8 v_3 \\
	\left( s_1 - 3 s_2 \right) s_7 &= 4 s_6 s_3
\end{align}

\subsection{Conversion of constants}

\begin{align}
s_{1} &= \kappa_1\\
s_{2} &= \kappa_2\\
s_{3} &= \kappa_3\\
s_{4} &= -\kappa_3\\
s_{5} &= \kappa_4\\
s_{6} &= \kappa_5\\
s_{7} &= \kappa_6\\
s_{8} &= \kappa_7\\
s_{9} &= \kappa_8\\
s_{10} &= 2  \kappa_7 - 6  \kappa_1 - 6  \kappa_2 - 2  \kappa_8\\
s_{11} &= \kappa_1 - \kappa_7 + \kappa_8\\
s_{12} &= (1/3)  \kappa_7 - \kappa_1 - (1/3)  \kappa_8\\
s_{13} &= \kappa_3\\
s_{14} &= -\kappa_3\\
s_{15} &= \kappa_4\\
s_{16} &= \kappa_5\\
s_{17} &= \kappa_6\\
s_{18} &= 3  \kappa_7 - 6  \kappa_1 - 6  \kappa_2 - 4  \kappa_8\\
s_{19} &= 2  \kappa_7 - 3  \kappa_1 - 3  \kappa_2 - 3  \kappa_8\\
s_{20} &= 2  \kappa_7 - 2  \kappa_8
\end{align}

\begin{align}
s_{21} &= \kappa_3\\
s_{22} &= -\kappa_3\\
s_{23} &= \kappa_9\\
s_{24} &= -\kappa_9\\
s_{25} &= -\kappa_4\\
s_{26} &= \kappa_6\\
s_{27} &= - 4  \kappa_5\\
s_{28} &= \kappa_{10}\\
s_{29} 
&= 
16  \kappa_1 + 16  \kappa_2 - (8/3)  \kappa_7 + (16/3)  \kappa_8 - \kappa_{10}\\
s_{30} 
&= 2  \kappa_7 - 12  \kappa_1 - 12  \kappa_2 - 4  \kappa_8 + \kappa_{10}\\
s_{31} 
&= 20  \kappa_1 + 20  \kappa_2 - 4  \kappa_7 + 8  \kappa_8 - \kappa_{10}\\
s_{32} &= \kappa_11\\
s_{33} &= \kappa_11\\
s_{34} &= 8  \kappa_1 + 8  \kappa_2 - (4/3)  \kappa_7 + (8/3)  \kappa_8\\
s_{35} &= -(4/3)  \kappa_1 - (4/3)  \kappa_2 + 2/3  \kappa_7 - 8/9  \kappa_8\\
s_{36} 
&= 2  \kappa_7 - 12  \kappa_1 - 12  \kappa_2 - 4  \kappa_8 + \kappa_{10}\\
s_{37} 
&= 20  \kappa_1 + 20  \kappa_2 - 4  \kappa_7 + 8  \kappa_8 - \kappa_{10}\\
s_{38} 
&= 4  \kappa_7 - 18  \kappa_1 - 18  \kappa_2 - 8  \kappa_8 + \kappa_{10}\\
s_{39} 
&= 
22  \kappa_1 + 22  \kappa_2 - (16/3)  \kappa_7 
+ (32/3)  \kappa_8 - \kappa_{10}\\
s_{40} &= \kappa_{11}\\
s_{41} &= \kappa_{11}\\
s_{42} &= 4  \kappa_1 + 4  \kappa_2 - (4/3)  \kappa_7 + (8/3)  \kappa_8\\
s_{43} 
&= -(2/3)  \kappa_1 - (2/3)  \kappa_2 + (4/9)  \kappa_7 - (2/3)  \kappa_8\\
s_{44} &= 24  \kappa_1 + 24  \kappa_2 - 4  \kappa_7 + 8  \kappa_8\\
s_{45} &= 12  \kappa_1 + 12  \kappa_2 - 4  \kappa_7 + 8  \kappa_8\\
s_{46} &= (4/3)  \kappa_7 - (4/3)  \kappa_8
\end{align}

\begin{align}
t_{1} &=-2\kappa_1 - 3\kappa_2 + \kappa_7 - \kappa_8\\
t_{2} &=-3\kappa_2 + \kappa_7 - \kappa_8 + \kappa_9\\
t_{3} &=\kappa_1\\
t_{4} &=-3\kappa_2 + \kappa_9\\
t_{5} &=\kappa_3\\
t_{6} &=-\kappa_3\\
t_{7} &=2\kappa_1 + 6\kappa_2 - 2\kappa_7 + 2\kappa_8 - \kappa_9\\
t_{8} &=\kappa_4\\
t_{9} &=2\kappa_4\\
t_{10} &=2\kappa_4\\
t_{11} &=\kappa_5\\
t_{12} &=\kappa_5\\
t_{13} &=\kappa_6\\
t_{14} &=\kappa_1\\
t_{15} &=-3\kappa_2 + \kappa_9\\
t_{16} &=\kappa_1 - \kappa_7 + \kappa_8\\
t_{17} &=3\kappa_1 - \kappa_7 + \kappa_8 + \kappa_9\\
t_{18} &=\kappa_3\\
t_{19} &=-\kappa_3\\
t_{20} &=-4\kappa_1 + 2\kappa_7 - 2\kappa_8 - \kappa_9
\end{align}

\begin{align}
t_{21} &=\kappa_4\\
t_{22} &=2\kappa_4\\
t_{23} &=2\kappa_4\\
t_{24} &=\kappa_5\\
t_{25} &=\kappa_5\\
t_{26} &=\kappa_6\\
t_{27} &=\kappa_3\\
t_{28} &=-\kappa_3\\
t_{29} &=\kappa_3\\
t_{30} &=-\kappa_3\\
t_{31} &=\kappa_9\\
t_{32} &=4\kappa_1 - 12\kappa_2 + 3\kappa_9\\
t_{33} &=-2\kappa_1 - 6\kappa_2 + 2\kappa_7 - 2\kappa_8 + \kappa_9\\
t_{34} &=4\kappa_1 - 2\kappa_7 + 2\kappa_8 + \kappa_9\\
t_{35} &=-\kappa_4\\
t_{36} &=-\kappa_4\\
t_{37} &=8\kappa_4\\
t_{38} &=\kappa_6\\
t_{39} &=\kappa_6\\
t_{40} &=-4\kappa_5
\end{align}

\begin{align}
	v_{1} &= -4\kappa_1 - 6\kappa_2 + 2\kappa_7 - 2\kappa_8\\
v_{2} &= \kappa_9/2\\
v_{3} &= 2\kappa_3\\
v_{4} &= -2\kappa_3\\
v_{5} &= 2\kappa_1 + 6\kappa_2 - 2\kappa_7 + 2\kappa_8 - \kappa_9\\
v_{6} &= 2\kappa_4\\
v_{7} &= 2\kappa_4\\
v_{8} &= 2\kappa_5\\
v_{9} &= 2\kappa_6\\
v_{10} &= -6\kappa_1 - 6\kappa_2 + 2\kappa_7 - 2\kappa_8\\
v_{11} &= 2\kappa_1\\
v_{12} &= \kappa_9/2\\
v_{13} &= 2\kappa_3\\
v_{14} &= -2\kappa_3\\
v_{15} &= -4\kappa_1 + 2\kappa_7 - 2\kappa_8 - \kappa_9\\
v_{16} &= 2\kappa_4\\
v_{17} &= 2\kappa_4\\
v_{18} &= 2\kappa_5\\
v_{19} &= 2\kappa_6\\
v_{20} &= 2\kappa_7 - 2\kappa_8
\end{align}

\begin{align}
v_{21} &= 2\kappa_3\\
v_{22} &= -2\kappa_3\\
v_{23} &= 2\kappa_9\\
v_{24} &= 2\kappa_1 - 6\kappa_2\\
v_{25} &= -2\kappa_1 - 6\kappa_2 + 2\kappa_7 - 2\kappa_8 + \kappa_9\\
v_{26} &= -2\kappa_4\\
v_{27} &= 8\kappa_4\\
v_{28} &= 2\kappa_6\\
v_{29} &= -8\kappa_5\\
v_{30} &= -6\kappa_1 - 6\kappa_2 + 4\kappa_7 - 4\kappa_8\\
v_{31} &= -6\kappa_1 - 6\kappa_2 + 2\kappa_7 - 2\kappa_8\\
v_{32} &= -6\kappa_1 - 6\kappa_2\\
v_{33} &= 6\kappa_1 + 6\kappa_2 - 4\kappa_7 + 4\kappa_8
\end{align}


\end{document}


















