\documentclass[11pt]{article}

\usepackage{amsmath}

\begin{document}

\title{The Generation of Waves in Gravitational Fields by Orbiting Charges}

I present a case study of moving matter that generates gravitational  waves. The system I study consists of two charged point masses, slowly orbiting each other due to electromagnetic attraction.

\section{The 3+1 Split}

To study my cases, I use the screen manifold formalism.

There are two relevant bases of the tangent space: the orthogonal basis $\{T, e_\alpha\}$ and the foliation basis $\{\dot{X} , \tilde{e}_\alpha\}$, with the corresponding cotangent space bases $\{ n, \epsilon^\alpha\}$ and $\{\kappa, \tilde{\epsilon}^\alpha \}$ respectively. These bases are related by

\begin{align}	
	\dot{X} &= N T + N^\alpha e_\alpha\\
	\tilde{e}_\alpha &= e_\alpha
\end{align}

and

\begin{align}	
	\kappa &= \frac{1}{N} n \\
	\tilde{\epsilon}_\alpha &= \epsilon_\alpha - \frac{1}{N} N^\alpha n
\end{align}

Formulas to derive / explain / proove:

\begin{align}
	\left[ e_{\alpha}, e_{\beta}\right] &= 0 \\
	\dot{T}^{A} &= \left( \mathcal{L}_{\dot{X}} T \right)^A
\end{align}

This section should contain a recipe which explains how to switch from the spacetime formalism to the screen manifold formalism for a generic quantity $T$ which comes with an action $S_T$:
\begin{enumerate}
	\item Determine the projections of $T$ on the screen manifold
	\item Determine the screen manifold action for those projections by expressing the spacetime action in foliation coordinates, and in terms of the projections
	\item From the action, calculate the Hamiltonian that generates the dynamics of the projections on the screen manifold with respect to the embedding parameter
\end{enumerate}


\section{Constructive Gravity}

The appropriate gravitational theory for a given matter theory depends on the causal structure (i.e. the principal polynomial $P$) of that matter theory, and is obtained by insisting on consistent dynamics. 

$P$ is the only relevant geometric object; the Legendre map $L$, which maps vectors to covectors, is obtained from $P$ by

\begin{equation}
	L^a \left( k \right) = \frac{1}{\deg{P}}\frac{\mathrm{D}^a P \left( k \right)}{P\left(k\right)}
\end{equation}

The universal action of massive point particles is

\begin{equation}
	S = m \int \mathrm{d}\tau \left(P \left( L^{-1} \left( \frac{\mathrm{d} \gamma\left(\tau \right)}{\mathrm{d}\tau} \right) \right)\right)^{\frac{1}{\deg{P}}}
\end{equation}

\section{Orbiting Charges in Metric Spacetime}

In this section, I firstly derive the Hamiltonians of the electromagnetic field and of $N$ charged point particles. Then, I determine the motion of two charged point particles which interact via the electromagnetic field. Finally, I use the linearised equations of metric gravity to determine the gravitational waves generated by the matter system. 

Metric specific formulae:

\begin{align}
	g^{00} &= 1 \\
	g^{\alpha 0} &= 0 \\
	g^{\alpha \beta} &:= g_{(3)}^{\alpha \beta}  := - \gamma^{\alpha \beta} + \varphi^{\alpha \beta}\\
	g_{\alpha \beta} &= - \gamma^{\alpha \beta} - \varphi^{\alpha \beta}
\end{align}

\subsection{The Electromagnetic field}

This subsections contains the derivation of the Hamiltonian of the electromagnetic field. The laws of Maxwells electromagnetism are condensed in the action

\begin{align}
	S &= \frac{1}{16 \pi} \int \mathrm{d}^4 x \left( - \det{g^{\cdot \cdot}} \right)^{-\frac{1}{2}} F_{ab} F_{cd} g^{ac} g^{bd}\\
	F_{ab} &:= \partial_a A_b - \partial_b A_a 
\end{align}

which can be found in the book of Landau and Lifshitz.

On the screen manifold, the spacetime object $A_a$ decomposes into

\begin{align}
	\phi &:= T^a A_a\\
	A_{\alpha} &:= e^a_{\alpha} A_a
\end{align}

The derivative with respect to the embedding parameter $t$ is 

\begin{align}
	\dot{A}_{a} &:= \left( \mathcal{L}_{\dot{X}} A \right)_a\\
	\dot{\phi} &:= T^a \dot{A}_{a}\\
	\dot{A}_{\alpha} &:= e^a_{\alpha} \dot{A}_{a}
\end{align}


Now, I express $S$ in terms of fields $\phi$ and $A_{\alpha}$ on the screen manifold, their derivatives with respect to $y_\alpha$ and $t$, and lapse $N$ and shift $N^{\alpha}$. I find

\begin{align}
	F_{\alpha \beta} := e^a_{\alpha} e^b_{\beta} F_{a b} = F^{(3)}_{\alpha \beta}
\end{align}

where $F^{(3)}_{\alpha \beta} := \partial_{\alpha} A_{\beta} - \partial_{\beta} A_{\alpha}  $ and I used that $\left[ e_{\alpha}, e_{\beta}\right] = 0 $, and

\begin{align}
	F_{0 \beta} &:= T^a e^b_{\beta} F_{a b} = \frac{1}{N} \left( \dot{A}_{\beta} - \partial_{\beta} \left( N \phi +  N^{\alpha} A_{\alpha} \right) - N^{\alpha}  F^{(3)}_{\alpha \beta} \right)
\end{align}

using $ T = \frac{1}{N} \left( \dot{X} - N^\alpha e_{\alpha} \right)$. Kuchar  and Stone grant that 

\begin{align}
	\mathrm{d}^4 x \left( - \det{g^{\cdot \cdot}} \right)^{-\frac{1}{2}} = \mathrm{d}t \, \mathrm{d}^3 y N \left( - \det{g_{(3)}^{\cdot \cdot}} \right)^{-\frac{1}{2}}
\end{align}
 
By substituting the above expressions in $S$, I obtain an action for the fields $\phi (t, y) $ and $A_{\alpha} (t, y) $ on the screen manifold. 

\begin{align}
	S = \frac{1}{16 \pi} \int \mathrm{d}t \int \mathrm{d}^3 y \, \,
	&N \left( - \det{g_{(3)}^{\cdot \cdot}} \right)^{-\frac{1}{2}} \\ 
	\times 
	&\left(
	 2 g_{(3)}^{\alpha \beta} F_{\alpha 0 } F_{\beta 0 }  
	 + g_{(3)}^{\alpha \beta} g_{(3)}^{\gamma \delta } F^{(3)}_{\alpha \gamma}  F^{(3)}_{\beta \delta} 
	\right)
\end{align}

The embedding parameter $t$ controls the dynamics on the screen manifold and serves as the time parameter for the Hamiltonian treatment; the canonical momenta are 

\begin{align}
	\Pi^{\alpha} := \frac{\partial L}{\partial \dot{A}_{\alpha}} = \frac{1}{4\pi} \left( - \det{g_{(3)}^{\cdot \cdot}} \right)^{-\frac{1}{2}} 
	 g_{(3)}^{\alpha \beta} F_{0 \beta}
\end{align}

and the Hamiltonian is

\begin{align}
	H &:= \Pi^{\alpha} \dot{A}_{\alpha} - L = N \mathcal{H} + N^\alpha \mathcal{D}_{\alpha}\\
\end{align}

with 

\begin{equation}
\begin{split}
	\mathcal{H} :=  
	2 \pi \left( - \det{g_{(3)}^{\cdot \cdot}} \right)^{\frac{1}{2}} 
	& g_{\alpha \beta}^{(3)} \Pi^{\alpha} \Pi^{\beta} \\
	 - &\frac{1}{16 \pi} \left( - \det{g_{(3)}^{\cdot \cdot}} \right)^{- \frac{1}{2}} g_{(3)}^{\alpha \beta} g_{(3)}^{\gamma \delta } F^{(3)}_{\alpha \gamma}  F^{(3)}_{\beta \delta} 
	- \phi \partial_{\alpha} \Pi^{\alpha} 
\end{split}
\end{equation}

\begin{align}
	\mathcal{D}_{\alpha} &:= \Pi^{\beta} F^{(3)}_{\alpha \beta} -  A_{\alpha} 
	\partial_{\beta} \Pi^{\beta}
\end{align}


For scenarios including only weak gravitational fields, I obtain a decent approximation to this Hamiltonian by substituting  $g_{(3)}^{\alpha \beta}  := - \gamma^{\alpha \beta} + \varphi^{\alpha \beta}$  and  $ N = 1 + A $, and then expanding the Hamiltonian to first order. I arrive at

\begin{equation}
\begin{split}
	H :=  
	& - \left( 1 + A \right) 
	\left[
	2 \pi \gamma_{\alpha \beta} \Pi^{\alpha} \Pi^{\beta}
	 + \frac{1}{8 \pi} \gamma_{\alpha \beta} H^{\alpha} H^{\beta}
	 + \phi \partial_{\alpha} \Pi^{\alpha}
	 \right]\\
	 & -  \varphi_{\mu \nu} 
	 \left( \delta_{\alpha}^{\mu}  \delta_{\beta}^{\nu} -  \frac{1}{2} \gamma^{\mu \nu } \gamma_{\alpha \beta} \right)
	 \left[
	 2 \pi  \Pi^{\alpha} \Pi^{\beta}
	 + \frac{1}{8 \pi} H^{\alpha} H^{\beta}
	 \right] \\
	 & + N^{\mu} \left[ 
	 \epsilon_{\mu \alpha \beta} \Pi^{\alpha} H^{\beta} -  A_{\mu} \partial_\alpha \Pi^\alpha
	 \right]
\end{split}
\end{equation}



\subsection{Point Particles}

In relativistic theory, a point particle corresponds to a curve $\gamma$ in spacetime. To switch to the screen manifold formalism - where the particle is represented by a position $\lambda (t) $ which changes with the foliation parameter $t$ - I parametrise $\gamma$ with the foliation parameter $t$ and decompose the particles velocity (i.e. the vector tangent to the parametrised curve) according to

\begin{equation}
	\frac{\mathrm{d} \gamma (t)}{\mathrm{d} t } =: \dot{X} +  e_\alpha v^\alpha
\end{equation}

This decomposition is defined such that $v = 	\frac{\mathrm{d} \lambda (t)}{\mathrm{d} t } := \dot{\lambda} (t) $, so if $ v^\alpha = 0 $ the particle  moves with the embedding e.g. remains at the same spot on the screen manifold.

The next step towards the autonomous screen manifold formalism requires to express the action that governs the dynamics of the particle solely in screen manifold quantities. For metric geometry, the universal action for a point mass in the presence of the electromagnetic field reads

\begin{equation}
	S = \int \mathrm{d}\tau  \, \, 
	m \left( 
	g_{a b}\left( \gamma \right)
	\frac{\mathrm{d} \gamma^a\left( \tau \right) }{\mathrm{d}\tau}
	\frac{\mathrm{d} \gamma^b\left( \tau \right) }{\mathrm{d}\tau} \right)^{\frac{1}{2}}
	+ e A_a \left( \gamma  \right) 
	\frac{\mathrm{d} \gamma^a\left( \tau \right) }{\mathrm{d}\tau} 
\end{equation}

Now parametrise $\gamma$ with $t$ and use the foliation frame to obtain the screen manifold action

\begin{equation}
\begin{split}
	S = \int \mathrm{d}\tau  \, \, 
	& m \left( 
	N^2 + g^{(3)}_{\alpha \beta} N^{\alpha} N^{\beta}
	+ 2 g^{(3)}_{\alpha \beta} v^\alpha N^\beta + g^{(3)}_{\alpha \beta} v^\alpha v^\beta 
	\right)^{\frac{1}{2}} \\
	+ \, & e \left( N \phi + N^\alpha A_\alpha + v^\alpha A_\alpha \right)
\end{split}
\end{equation}

To calculate the Hamiltonian of the point mass, I proceed slightly differently than above: this time, I firstly linearise the action and then execute the Legendre transform with the benefit that the inversion in favour of the velocity must not be done exactly, but pertubatively.

The linearised action reads

\begin{equation}
\begin{split}
	S = \int \mathrm{d}\tau  \, \, 
	& m \left(
	\left( 1 - v^2 \right)^{\frac{1}{2}}
	+ \left( 1 - v^2 \right)^{- \frac{1}{2}}
	\left( A - \gamma_{\alpha \beta} N^\beta v^\alpha - \frac{1}{2} \varphi_{\alpha \beta} v^\alpha v^\beta \right)
	\right)\\
	+ \, \, & e \left( \phi + v^\alpha A_\alpha + A \phi + N^\alpha A_\alpha \right) + \mathcal{O} \left(\varphi^2 \right)
\end{split}
\end{equation}

The canonical momentum of the point mass is

\begin{equation}
	p_\alpha := \frac{ \partial L}{\partial v^\alpha}
	= -m \frac{v^\alpha}{\sqrt{1- v^2}} + e A_\alpha	 
	+ \mathcal{O} \left(\varphi^2 \right)
\end{equation}

I define $k_\alpha = p_\alpha - e A_\alpha$, expand $v^\alpha = v^\alpha_0 + v^\alpha_1 + \dots $ in orders of $\varphi$ and, by virtue of

\begin{align}
	k_\alpha &= -m \frac{v^\alpha}{\sqrt{1- v^2}} 
	 &\Leftrightarrow 
	 v_\alpha &= - \frac{k_\alpha}{\sqrt{k^2 + m^2}} 
\end{align}

I obtain

\begin{equation}
	v_\alpha = - \frac{k_\alpha}{E_k} + \mathcal{O} \left(\varphi^2 \right)
\end{equation}

where $E_k := \sqrt{k^2+ m^2}$. Accommodatingly, knowledge of the zeroth order of $v^\alpha$ suffices to obtain the linearised Hamiltonian throught the Legendre transform:

\begin{equation}
	H := p_\alpha v^\alpha - L  
	= - \left( 1 + A \right) \left( E_k + e \phi \right)
	- N^\alpha p_\alpha + \frac{1}{2 E_k} \varphi_{\alpha \beta} k^\alpha k^\beta + \mathcal{O} \left(\varphi^2 \right)
\end{equation}

where all occurring fields are evaluated at the position $\lambda$ of the particle

\subsection{The Motion of Two Charged Point Masses}

\subsection{The Generation of Gravitational Waves in Metric Spacetime}

\section{Orbiting Charges in Weakly Birefringent Spacetime}

\end{document}
